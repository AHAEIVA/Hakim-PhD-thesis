%\documentclass[a4paper, 10 pt, conference]{ieeeconf_ilker}
%\IEEEoverridecommandlockouts
%\overrideIEEEmargins

\documentclass[letterpaper, 10 pt, conference]{ieeetran}  % Comment this line out if you need a4paper
                                                   
%\documentclass[a4paper, 10pt, conference]{ieeeconf}      % Use this line for a4 paper

\IEEEoverridecommandlockouts                              % This command is only needed if 
                                                          % you want to use the \thanks command

\overrideIEEEmargins                                      % Needed to meet printer requirements.
                                                          % you want to use the \thanks command
                                                          
% \overrideIEEEmargins                                      % Needed to meet printer requirements.
\pdfminorversion=4 

% setup page to suit conference specification using fancyhdr
%\usepackage{fancyhdr}
%\setlength{\paperwidth}{215.9mm}
%\setlength{\hoffset}{-9.7mm}
%\setlength{\oddsidemargin}{0mm}
%\setlength{\textwidth}{184.3mm}
%\setlength{\columnsep}{6.3mm}
%\setlength{\marginparsep}{0mm}
%\setlength{\marginparwidth}{0mm}

%\setlength{\paperheight}{279.4mm}
%\setlength{\voffset}{-7.4mm}
%\setlength{\topmargin}{19mm}
%\setlength{\headheight}{0mm}
%\setlength{\headsep}{0mm}
%\setlength{\topskip}{0mm}
%\setlength{\textheight}{235.2mm}
%\setlength{\footskip}{12.4mm}

%\setlength{\parindent}{1pc}


% correct bad hyphenation here
\hyphenation{op-tical net-works semi-conduc-tor}
\usepackage{blindtext}
\usepackage{tikz}
\usepackage[utf8]{inputenc}
\usepackage[english]{babel}
\usepackage{subcaption}  % Add this in the preamble if not already included

\usepackage[ruled,vlined,linesnumbered]{algorithm2e}
\usetikzlibrary{positioning}
\usetikzlibrary{shapes.geometric}
\usetikzlibrary{arrows.meta}
\usepackage{subfig}
\usepackage[nolist]{acronym}
\usepackage[table, dvipsnames]{xcolor}
\usepackage[pdftex]{graphicx} % graphics package
\usepackage{graphicx}
\usepackage[colorinlistoftodos]{todonotes}
%\usepackage[lofdepth,lotdepth,caption=false]{subfig} % subplot package
\usepackage{graphicx}
\usepackage{caption}
\usepackage{algorithm}
\usepackage{algpseudocode}
\captionsetup{compatibility=false}
\usepackage{subcaption}
\usepackage{tikz}
\usetikzlibrary{shapes.geometric, arrows}
\usepackage{array} % for defining custom column types
\usepackage{amssymb} %we added the package to the document
\usepackage{algorithm}
\usepackage{algpseudocode}
\usepackage{amsmath}  %

\usepackage{algorithm}
\usepackage{algorithmicx}
\usepackage{algpseudocode}
\usepackage{amsmath}
\usepackage[small]{caption} % Reduce text size

\usepackage{booktabs}
\graphicspath{{figures/}} % declare the path where graphic files are
\DeclareGraphicsExtensions{.pdf,.jpg,.png} % grafic files extensions
\usepackage{amsmath} % math package for split
\usepackage{amsfonts} % math package for fonts
\usepackage{mathtools} % math package for tools like prescript
\usepackage{tikz} % figure package
\usetikzlibrary{shapes,arrows,fit,calc,positioning,automata,backgrounds}
%\usepackage{dblfloatfix} % figure package for positioning figures on the bottom
\usepackage{color} % colors package
\usepackage{multirow} % multiple rows in table
\usepackage{hhline} % separator in 
\usepackage{booktabs}
\usepackage[bookmarks=false, breaklinks]{hyperref} % hyperlink package
\usepackage{enumerate} % enumerate package
\usepackage{epstopdf} % package to include eps. files
\usepackage{dsfont}
\usepackage{siunitx} % package for international system units
\usepackage{cite}   % for compact citations, i.e. [1, 2]
\usepackage{diagbox}    % for making slash cell in a table
\usepackage{pdfpages}   
\usepackage{breqn}
\usepackage{svg}
\usepackage{multicol}
\svgpath{{images/svg/}} % <- using \svgpath to avoid warning
\colorlet{Green1}{green!90!}
\colorlet{Green2}{green!60!}
\colorlet{Green3}{green!40!}
\colorlet{Green4}{green!20!}
\colorlet{Green5}{green!10!}

% Used in order to break URL into smaller sections when seeing / or -
\usepackage{url}
\def\UrlBreaks{\do\/\do-}
\usepackage{breakurl}

%% ORCID
\RequirePackage{tikz} % For \foreach used for Orcid icon
% Make Orcid icon
\newcommand{\orcidicon}{\includegraphics[width=0.32cm]{images/orcid/logo-orcid.eps}}

% Define link and button for each author
\foreach \x in {A, ..., Z}{%
\expandafter\xdef\csname orcid\x\endcsname{\noexpand\href{https://orcid.org/\csname orcidauthor\x\endcsname}{\noexpand\orcidicon}}
}

\newcommand{\orcidauthorA}{0000-0001-5387-7613} % Add \orcidA{} behind the author's Hakim
\newcommand{\orcidauthorB}{0000-0003-3581-9481} % Add \orcidB{} behind the author's Olaya
\newcommand{\orcidauthorC}{0000-0002-1609-5783} % Add \orcidC{} behind the author's name  Halil
\newcommand{\orcidauthorD}{0000-0002-7742-6996} % Add \orcidD{} behind the author's name  Jonas
\newcommand{\orcidauthorE}{0009-0002-0445-8126} % Add \orcidE{} behind the author's name  Yury
\newcommand{\orcidauthorF}{0000-0002-7143-8777} % Add \orcidF{} behind the author's name  Erdal


% TikZ style definitions
\tikzstyle{startstop} = [rectangle, rounded corners, minimum width=3cm, minimum height=1cm, text centered, draw=black, fill=blue!20]
\tikzstyle{process} = [rectangle, minimum width=3cm, minimum height=1cm, text centered, draw=black, fill=green!20]
\tikzstyle{decision} = [diamond, minimum width=3cm, minimum height=1cm, text centered, draw=black, fill=red!20]
\tikzstyle{arrow} = [thick,-{Latex[length=3mm, width=2mm]}]



\definecolor{Bookcolor}{HTML}{00F9DE}
\definecolor{darkgreen}{rgb}{0.0, 0.5, 0.0}
\definecolor{gray}{gray}{0.9}
\definecolor{lightgray}{rgb}{0.86, 0.86, 0.86}

\makeatletter
\def\@citex[#1]#2{\leavevmode
\let\@citea\@empty
\@cite{\@for\@citeb:=#2\do
{\@citea\def\@citea{,\penalty\@m\ }%
\edef\@citeb{\expandafter\@firstofone\@citeb\@empty}%
\if@filesw\immediate\write\@auxout{\string\citation{\@citeb}}\fi
\@ifundefined{b@\@citeb}{\hbox{\reset@font\bfseries ?}%
\G@refundefinedtrue
\@latex@warning
{Citation `\@citeb' on page \thepage \space undefined}}%
{\@cite@ofmt{\csname b@\@citeb\endcsname}}}}{#1}}
\makeatother


\newcommand{\HP}[1]{\textcolor{blue}{[HP: #1]}}
\newcommand{\HIU}[1]{\textcolor{red}{[HIU: #1]}}
\newcommand{\KA}[1]{\textcolor{orange}{[KA: #1]}}
\newcommand{\EK}[1]{\textcolor{magenta}{[EK: #1]}}
\newcommand{\JFS}[1]{\textcolor{darkgreen}{[JFS: #1]}}
\newcommand{\note}[1]{\textcolor{red}{[NOTE: #1]}}

\DeclareMathOperator*{\minimize}{{min}}
\DeclareMathOperator*{\maximize}{{max}}

\newcommand{\bm}[1]{\mathbf{#1}}

\begin{acronym}
    % \acro{CAMAP}{conflict-aware multi-agent planner}

    \acro{MPC}{model predictive control}
    \acro{GP}{Gaussian process}
    \acro{AUV}{autonomous underwater vehicle}
    \acro{MHE}{moving horizon estimator}
    \acro{EKF}{extended kalman filter}
    \acro{ROV}{remotely operated vehicle}
    \acro{TSP}{traveling salesman problem}
    \acro{IMU}{inertial measurement unit}
    \acro{DVL}{doppler velocity log}
    \acro{NED}{North East Down}
    \acro{RTI}{real-time iteration}
    \acro{OCP}{optimal control problem}
    \acro{LP}{linear program}
    \acro{MIP}{mixed integer program}
    \acro{EA-MPC}{entanglement aware model predictive control}
    \acro{TSDF}{ truncated signed distance field}
    \acro{SDF}{ signed distance field}
    \acro{CPP}{ coverage path planner}
    \acro{OEA-PP}{online entanglement aware path planner}
    \acro{OMPL}{open motion planning}
        \acroplural{ROV}[ROVs]{remotely operated vehicles}

\acro{REACT}{Real-time Entanglement-Aware Coverage
Path Planning for Tethered Underwater Vehicles}

  % Ensures ROVs is treated the same as ROV
}


\end{acronym}




\begin{document}

%

%\title{Online Entanglement-Aware Path Planning for Safe Autonomous Underwater Vehicle Inspection}
\title{REACT: Real-time Entanglement-Aware Coverage Path Planning for Tethered Underwater Vehicles}

% author names and affiliations
% use a multiple column layout for up to three different affiliations
%\author{Abdelhakim Amer, Mohit Mehindratta, Yury Brodskiy and Erdal Kayacan
\author{
% <-this % stops a space
\thanks{A. Amer is with the Artificial Intelligence in Robotics Laboratory (AiR Lab), Department of Electrical and Computer Engineering, Aarhus University, 8000 Aarhus C, Denmark {\tt\small \{abdelhakim\} at ece.au.dk}.    Mohit Mehndiratta is with GIM Robotics, Espoo, Finland, {\tt\small \{mohit.mehndiratta\} at gimrobotics.fi}.
     Y. Brodskiy is with EIVA a/s, 8660 Skanderborg, Denmark. {\tt\small \{ybr\} at eiva.com.}
    E. Kayacan is with the Automatic Control Group, Department of Electrical Engineering and Information Technology, Paderborn University, Paderborn, Germany. {\tt\small \{erdal.kayacan\} at uni-paderborn.de}.}%
}

\maketitle
\begin{abstract}
Autonomous inspection of complex underwater structures with tethered Underwater Vehicles is often hindered by the risk of tether entanglement. We propose REACT (Real-time Entanglement-Aware Coverage Path Planning for Tethered Underwater Vehicles), a framework designed to overcome this limitation. REACT comprises a fast, geometry-based tether model using the signed distance field (SDF) map for accurate, real-time simulation of taut tether configurations around arbitrary structures. This model enables an efficient online replanning strategy within REACT that enforces a maximum tether length constraint, actively preventing entanglement. By integrating REACT into a coverage path planning framework, we achieve safe and optimal inspection paths, previously challenging due to tether constraints. To ensure safe and precise path following, a model predictive controller (MPC) is employed, accounting for system dynamics while maintaining the desired trajectory. The complete REACT framework's efficacy is validated in a pipe inspection scenario, demonstrating safe, entanglement-free navigation and full coverage inspection.




% What is the problem this paper is trying to solve?

% What is the approach?


% What does the results show?

Autonomous underwater vehicles (AUVs) present several challenges due to the complex and simultaneous interplay of various factors, including but not limited to unmodeled dynamics, highly nonlinear behaviour, inter-couplings, communication delays, and environmental disturbances. In particular, environmental disturbances degrade trajectory tracking performance for model-based controllers, e.g., model predictive control (MPC) algorithms. Data-driven methods such as Gaussian process (GP) are effective at learning disturbances in real-time, however, the underlying offline hyperparameter tuning process limits their overall effectiveness. To overcome this limitation, we propose a novel dynamic forgetting Gaussian process (DF-GP) methodology that compensates for operational disturbances, thus circumventing the need for hyperparameter retuning. In essence, the proposed method optimally combines the predictions of individual GPs -- designed with handcrafted forgetting factors --, rendering precise disturbance estimation of varying timescales. What is more, the predicted disturbances update the model parameters in MPC, facilitating a learning-based control framework that ensures accurate tracking performance in different underwater scenarios. Rigorous simulation and real-world experiments demonstrate the efficiency and efficacy of the proposed framework. The results show a 25\% improvement in disturbance estimation and tracking performance, demonstrating that the proposed framework outperforms its direct competitors.%
\end{abstract}

\begin{IEEEkeywords}
coverage path planning, autonomous underwater vehicle (AUV), model predictive control, signed distance field (SDF), tether modelling, tether constraints.
\end{IEEEkeywords}

\IEEEpeerreviewmaketitle

%\bstctlcite{IEEEexample:BSTcontrol}

% introduction and related work




\section{Introduction}
\label{sec:introduction}

% % Underwater Robot Applications:
%
In many applications, \Acfp{ROV} play a pivotal role, including surveying, inspection, and search-and-rescue operations, by improving efficiency and ensuring safety \cite{amer2023unav, amergp}.
%
% Why We Need Tether:
%
Most \acp{ROV} are tethered to ensure a continuous power supply during long-duration missions and to maintain reliable communication.
%
% Introducing Problem Definition:
%
However, the presence of a tether introduces complexities in path planning and control, as it poses a risk of entanglement with underwater objects such as flora, fauna, or structures being inspected.

This challenge restricts the applicability of many path planning algorithms originally designed for untethered drones and surface vehicles. For instance, numerous \ac{CPP} algorithms exist to compute distance-optimal paths for covering 3D structures \cite{bircher2015structural,feng2024fc, amer2023visual}. Additionally, exploration path planners are employed to determine the next viewpoints for mapping and exploring unknown terrains \cite{dang2020graph}. However, these methods do not account for entanglement with the surroundings and thus cannot be directly applied to tethered underwater robots.
%
% Entanglement Problem and Definition:
%
Entanglement occurs when the movement of the underwater vehicle is restricted due to physical interactions between the tether and objects in the environment. Tether can bend or loop around obstacles, limiting vehicle mobility. This work bridges a critical gap in automatic asset inspection with \ac{ROV} by proposing \ac{REACT}.

The contributions of this work can be summarized as follows:
\begin{itemize}
\item A fast online tether model that computes the tether configuration in real time using \ac{SDF} data of arbitrary underwater structures.
\item An efficient online replanning method that prevents entanglement by incorporating a maximum tether length constraint.
\item Integration of the proposed method into a coverage path planning framework with \ac{MPC} to enable optimal inspection of underwater structures while avoiding entanglement.

\item Demonstration of the complete framework in simulation, showcasing its ability to ensure safe and efficient underwater structure inspection.
\end{itemize}





%
\begin{figure}[t!]
	\centering	\includegraphics[width=0.7\linewidth]{figures/react_abstract.pdf}
	  \caption{\textcolor{black}{Overview of the \ac{REACT} inspection framework. In the offline phase, a \ac{SDF} map is generated from a point cloud, and the FC-Planner \cite{feng2024fc} computes an optimal waypoint sequence. In the online phase, a tether model $\mathbf{P}^{tether}(t)$ ensures the maximum tether length is maintained, while an \ac{MPC} controller applies the optimal wrench $\mathbf{u}(t)$ to the \ac{ROV}}.}
    \label{fig:abstract}
\end{figure}
%


The remainder of this paper is organized as follows: Section \ref{sec:related_work} reviews the state of the art. Section \ref{sec:framework} presents the path planning framework for underwater structure inspection with tether constraints. Section \ref{sec:tether_model} introduces the taut-tether model, and Section \ref{sec:planner} details the online entanglement-aware path planner. Experimental results are shown in Section \ref{sec:experiments}, followed by conclusions and future work in Section \ref{sec:conclusion}.





 
\section{State of the art}
\label{sec:related_work}

% Paragraph 1: Introduction to Tether Importance and Entanglement Definition
Tethers play a vital role in numerous mobile robotics applications, offering crucial power delivery and reliable data transmission. It enables extended operational periods and ensures consistent communication links for the robot. However, the risk of entanglement with obstacles represents a significant planning challenge. To systematically address this, a taxonomy of entanglement definitions has been presented in \cite{definitions}, cataloging existing interpretations from the literature and introducing new definitions, paving the way for developing new path planning strategies that account for entanglement.

% Paragraph 2: Early Approaches - 2D Offline Path Planning (Single Robot)
Initial efforts to develop entaglement-aware path planners often focused on two-dimensional environments and employed offline computation strategies. For instance, \cite{rov_mccammon} and \cite{mechsy2017novel} proposed methods for planning paths in 2D to cover a predefined set of waypoints while considering tether constraints. Notably, \cite{mechsy2017novel} formulated the path planning problem for a \ac{ROV} as a mixed-integer programming problem. Their approach first solves a \ac{TSP} to find an optimal waypoint sequence and then incorporates homotopic constraints during path generation to minimize the likelihood of tether entanglement. While effective for predefined scenarios, these offline methods lack the adaptability required for dynamic environments.

% Paragraph 3: Advancements - 2D Online Path Planning (Single Robot)
Addressing the need for real-time adaptability, subsequent research explored online path planning algorithms, still primarily in 2D. \cite{kim} introduced a homotopy-augmented topological approach combined with graph search techniques, allowing for dynamic adjustments to the path based on environmental perception. Similarly, \cite{withy} developed a hybrid A* variant that utilizes a modified tangent graph. This method efficiently plans curvature-constrained paths for tethered robots subject to winding angle constraints, demonstrating guarantees and providing simulation results for online entanglement avoidance. %These online planners represent a significant step towards real-time tether management.

% Paragraph 4: Scaling Complexity - Multi-Robot Systems in 2D
The complexity of tether management increases significantly when coordinating multiple robots. Several approaches have tackled this challenge in 2D. Early work by \cite{hert1996ties} laid groundwork in this area, followed by methods like \cite{zhang2019planning}. More recently, \cite{cao2023neptune} presented an efficient online path planner for multi-robot systems. Their method integrates a homotopy-based high-level planner with trajectory optimization and smoothing techniques to generate entanglement-free paths. However, despite its online capability, this approach remains constrained to 2D environments.

% Paragraph 5: Moving to Three Dimensions - 3D Path Planning (Single Robot)
Real-world applications frequently require navigation in 3D environments, such as with underwater robots. Consequently, work by  \cite{bhattacharya2012topological} and \cite{martinez2021optimization} explored topological aspects and optimization techniques for 3D tethered navigation. \cite{petit2022tape} specifically presented a 3D exploration path planner that incorporates explicit contact avoidance constraints for the tether, facilitating safer navigation for single tethered robots in complex three-dimensional spaces.

% Paragraph 6: 3D Multi-Robot Path Planning
 \cite{hert1999motion} extended their earlier 2D work to handle the increased complexity of 3D multi-robot scenarios. Further advancements by \cite{patil2023coordinating} and  \cite{cao2023path} introduced path planning strategies that explicitly consider the topological constraints imposed by multiple interacting tethers in 3D. While these methods advance the state-of-the-art in multi-robot coordination, they are generally designed for offline computation and are not suited for online path planning where real-time implementation is essential.

% Paragraph 7: Identifying the Research Gap
% Summarize the limitations of existing work: 
In summary, existing tether-aware planners often face limitations for practical online coverage path planning (CPP) in complex 3D settings. Many are too computationally intensive for real-time use \cite{mechsy2017novel, hert1999motion, patil2023coordinating, cao2023path}, lack integrated tether-aware CPP frameworks, or rely on simplifying assumptions like 2D environments or basic obstacle shapes \cite{kim, withy, cao2023neptune}, hindering generalization to real-world inspection tasks.
% Paragraph 8: Introducing the Proposed Solution (REACT) based on its key contributions
To address these limitations, we propose \ac{REACT}, a novel approach that enables real-time, entanglement-aware path planning in arbitrary 3-D environments.


% --- End of Literature Review Section ---













































% % Introduce tether in literature
% Tether considerations in autonomous robot planning has drawn increasingly attention in robotics literature due to their critical role in enabling robots to operate autonomously in unknown environments for a wide range of mobile robots. 
% % Entaglement definitions
% A taxonomy of entanglement definitions is presented in \cite{definitions} which catalogs existing definitions from the literature and also introduces six new definitions.  These definitions open the door to the design of new entanglement aware path planning approaches.

% % 2-D , offline
% Regarding the path planners, many path planning methods have been proposed to address tether entanglement in 2-D environments. \cite{rov_mccammon} and \cite{mechsy2017novel} propose 2-D offline path planning to cover a set a way-points. In particular, \cite{mechsy2017novel} formulate the path planning problem for a \ac{ROV} as a mixed-integer programming problem. Their method solves a \ac{TSP} to determine the optimal sequence of waypoint visits, while incorporating homotopic constraints to reduce the likelihood of tether entanglement.
% % 2-D , online
% Also in 2-D, but in an online fashion, a homotopy-augmented topological approach, combined with graph search techniques is proposed by \cite{kim}. In addition to offline methods, online path planners allow real-time entanglement avoidance \cite{kim}, \cite{withy}. A hybrid A* variant uses a modified tangent graph to efficiently plan curvature-constrained, tethered robot paths under winding angle constraints, with guarantees and simulation results provided \cite{withy}.
% % 2-D , Multirobot
% Other approaches, also in 2-D address the tether constraints problem for the multi-robot case in \cite{zhang2019planning}, \cite{hert1996ties}, and \cite{cao2023neptune}. An efficient online path planner is presented in \cite{cao2023neptune}, combining a homotopy-based high-level planner with trajectory optimization and smoothing to generate entanglement-free paths for multi-robot systems. However, this approach is limited to 2-D environments. 
% % 3-D , Multirobot
% Entanglement avoidance has also been investigated in 3-D environments, as demonstrated in \cite{petit2022tape}, \cite{martinez2021optimization}, and \cite{bhattacharya2012topological}. In particular, \cite{petit2022tape} presents a 3-D exploration path planner that incorporates contact avoidance constraints on the tether, enabling safe navigation for tethered robots in three-dimensional spaces. For multi-robot operations, \cite{hert1999motion} extends earlier work to handle the increased complexity of coordinating multiple tethered robots. Further advancements by \cite{patil2023coordinating} and \cite{cao2023path} introduce improved coordination strategies that consider topological constraints imposed by multiple tethers. However, these approaches are not designed for real-time, online path planning.

% % NICHE and summary of research gap 
% In summary, most existing planners suffer from limitations that hinder their use in practical, online coverage path planning applications. Many are too computationally intensive for real-time implementation, and there is a lack of coverage path planning frameworks that explicitly account for the tether during planning. Existing tether-aware methods often rely on simplifying assumptions, such as 2-D environments or simplistic obstacle shapes (e.g., circles or cylinders), and thus fail to generalize to complex, real-world environments—particularly in inspection tasks.
% % Introduce REACT
% To address these limitations, we propose \ac{REACT}, a novel approach that enables real-time, entanglement-aware path planning in arbitrary 3-D environments.

 


\section{Overall \ac{REACT}  Framework}
\label{sec:framework}
\ac{REACT} framework, as depicted in \ref{fig:abstract}, consists of two main components: an offline planning phase and an online execution phase. In the offline phase, the environment is provided as a point cloud, including the object to be inspected. A \ac{SDF} map is then generated from this point cloud using the nvblox library \cite{nvblox}. The extracted point cloud of the structure to be inspected is then processed by the FC-Planner \cite{feng2024fc} to compute an optimal ordered sequence of waypoints, to acheive a path that performs full inspection coverage, initially disregarding tether constraints.

%Online Phase
In the online phase, tether constraints are handled by a entanglement-aware re-planner that ensures that the maximum tether length is not exceeded due to entanglement. This is achieved using our developed tether model, which continuously updates based on the current state of the tether and the new position of the \ac{ROV}, denoted as $\textbf{p}_{rov}$. The online planner then provides the reference state to an \ac{MPC} controller, which computes and applies the optimal wrench (forces and torques) to the \ac{ROV}.



%%%%%%%%%%%%%%%%%%%%%
%%%%%%  Tether Figure  
%%%%%%%%%%%%%%%%%%%%%

\begin{figure*}[t!]
    \centering
    \includegraphics[width=1\linewidth]{Phd_thesis/figures/tether_model.pdf}
    \caption{Tether shortcutting during \ac{ROV} motion from \( t_1 \) to \( t_2 \). (1) Initial tether with new \ac{ROV} position \( \mathbf{p}_{\text{rov}} \) appended; (2) Successful shortcut from the end node; (3) Collision encountered when attempting further shortcutting, prompting move to the next node; (4) Another collision detected from the new node; (5) Successful shortcut from a subsequent node; (6) Final tether configuration (yellow) after applying all feasible shortcuts.}
    \label{fig:tether}
\end{figure*}
%%%%%%%%%%%%%%%%%%%%%%%%



%%%%%%%%%%%%%%%%%%%%%%%%%%%
%%%%%%%%%%%%%%%%%%%%%%%%%%%
%%%%%% Tether Model Section
%%%%%%%%%%%%%%%%%%%%%%%%%%%
%%%%%%%%%%%%%%%%%%%%%%%%%%%
\section{Tether Model}
\label{sec:tether_model}
In this section, we introduce a computationally efficient kinematic \ac{ROV} tether model. Inspired by the shortcutting algorithm used in the ropeRRT path planner \cite{roperrt}, which simplifies sampled trajectories in a manner analogous to a rope tightening around an obstacle. The key assumption in the proposed model is that we assume a geometry-based constraint where the tether remains taut and fully stretched at all times.

\subsection{Tether model description}
Let the tether path at time \( t \) be denoted by \( \mathbf{P}^{tether}(t) = \{ p_i(t) \}_{i=1}^{N} \), where each node \( p_i(t) \in \mathbb{R}^3 \) represents the position of the \( i \)-th node in 3D space at time \( t \), and \( N \) is the total number of nodes in the tether path. The position of the ROV at time \( t \), denoted by \( \mathbf{p}_{rov}(t) \in \mathbb{R}^3 \), is appended \( p_{N+1}(t) \) at the end of the tether path, ensuring that the ROV's position is included in the configuration. 

The proposed tether model iteratively optimizes the tether path via two main mechanisms: shortcutting and pulling. For each pair of nodes \( (p_i(t), p_j(t)) \) where \( i < j \), the algorithm attempts to shortcut the path segment between them. If the line of sight between \( p_i(t) \) and \( p_j(t) \) is collision-free, determined via a \ac{SDF} map \( \mathcal{M}_{sdf} \), the intermediate nodes are replaced with a straight segment sampled at a resolution \( \delta_n \).





%%%%%%%%%%%%%%%%%%%%%%%%%%%%%%%%%%%%%%%%%%
%%%%%%  Tether Model Algorithm  %%%%%%%%%
%%%%%%%%%%%%%%%%%%%%%%%%%%%%%%%%%%%%%%%%%%


\begin{algorithm}[H]
%\SetAlgoLined
\LinesNotNumbered  % Disable line numbers

\SetKwInOut{Input}{Require}
\SetKwInOut{Output}{Return}
\Input{
ROV position $\mathbf{p}_{rov}(t)$, 
Tether path $\mathbf{P}^{tether}(t)$, 
SDF map $\mathcal{M}_{sdf}$, 
parameters: $\delta_{n}$
}
\Output{Taut-tether at $t+1$ : $\mathbf{P}^{tether}(t+1)$}
\BlankLine

\tcp{Extend tether path to include the current ROV position as the endpoint}
Append $\mathbf{p}_{rov}(t)$ to $\mathbf{P}^{tether}(t).end()$\;

%\tcp{Iterative shortcutting path segments by skipping intermediate nodes}
\tcp{Iterative shortcutting}
\For{$i \gets len(\mathbf{P}^{tether}(t)) - 2$ \textbf{to} $1$}{
    \For{$j \gets i+1$ \textbf{to} $len(\mathbf{P}^{tether}(t)) - 1$}{
        \tcp{Check if shortcutting the path is valid between nodes $i$ and $j$}
        \If{checkshortcut($\mathbf{P}^{tether}(t), i, j$)}{
            \tcp{Replace intermediate nodes $i-j$ }
            ReplaceIntermediateNodes($\mathbf{P}^{tether}(t)$, $i$, $j$, $\delta_{n}$)\;
        }
        \Else{
            \If{not CheckLineOfSight($\mathcal{M}_{sdf}$, $\mathbf{P}^{tether}(t)$)}{
                \tcp{Line of sight is blocked by obstacle, stop shortcutting}
                \textbf{break}\;
            }
            \If{isInCollision($\mathcal{M}_{sdf}$, $\mathbf{P}^{tether}(t)[j]$)}{
                \tcp{Pull node towards tether end}
                PullNode($\mathbf{P}^{tether}(t)[j]$, $\mathbf{P}^{tether}(t).end()$, $\delta_{n}$)\;
            }
        }
    }
}
\Return{$\mathbf{P}^{tether}(t+1)$}\;
\caption{Taut-Tether Model}
\label{alg:tether_optimization}
\end{algorithm}



%%%%%%%%%%%%%%%%%%%%%%%%%%%
%%%%%%%%%%%%%%%%%%%%%%%%%%%
% Homotopy equivelance proof
%%%%%%%%%%%%%%%%%%%%%%%%%%%
%%%%%%%%%%%%%%%%%%%%%%%%%%%


% \subsection{Homotopic Equivalence of the Shortcutting-Based Tether Path}
% \label{sec:homotopy_proof}

% In this subsection, we demonstrate that the taut-tether path generated primarily through the shortcutting mechanism is homotopically equivalent to the Remotely Operated Vehicle's (ROV) actual trajectory. This argument focuses on the geometric simplification provided by shortcutting, assuming the tether behaves like a rope tightening around obstacles.

% \subsubsection{Preliminaries and Definitions}
% Let the ambient Euclidean space be $X = \mathbb{R}^3$. The static obstacle region, $O \subset X$, is a closed set. The \textbf{free configuration space} is defined as $C_{free} = X \setminus O$. A \textbf{path} in $C_{free}$ is a continuous function $\gamma: [0,1] \to C_{free}$.

% Two paths $\gamma_0, \gamma_1: [0,1] \to C_{free}$ sharing the same endpoints (i.e., $\gamma_0(0)=\gamma_1(0)$ and $\gamma_0(1)=\gamma_1(1)$) are said to be \textbf{homotopic in $C_{free}$} (denoted $\gamma_0 \sim \gamma_1$) if there exists a continuous function $H: [0,1] \times [0,1] \to C_{free}$, called a \textbf{homotopy}, such that:
% \begin{itemize}
%     \item $H(s,0) = \gamma_0(s)$ for all $s \in [0,1]$,
%     \item $H(s,1) = \gamma_1(s)$ for all $s \in [0,1]$,
%     \item $H(0,\tau) = \gamma_0(0)$ for all $\tau \in [0,1]$ (startpoints fixed),
%     \item $H(1,\tau) = \gamma_0(1)$ for all $\tau \in [0,1]$ (endpoints fixed).
% \end{itemize}
% Let $p_A \in C_{free}$ denote the fixed anchor point of the tether and $p_R(t) \in C_{free}$ be the ROV's position at time $t$. The actual continuous trajectory of the ROV up to time $t$ is a path $\Gamma_{ROV}: [0,1] \to C_{free}$, with $\Gamma_{ROV}(0) = p_A$ (assuming the path starts at the anchor for simplicity) and $\Gamma_{ROV}(1) = p_R(t)$.

% A \textbf{piecewise linear (PL) path}, $\mathbf{P}$, connecting $p_A$ to $p_R(t)$ is defined by an ordered sequence of $N+1$ vertices $(v_0, v_1, \ldots, v_N)$ where $v_0=p_A$, $v_N=p_R(t)$, and each $v_k \in C_{free}$. The path consists of line segments $L(v_k, v_{k+1}) = \{ (1-s)v_k + s v_{k+1} \mid s \in [0,1] \}$. For $\mathbf{P}$ to be a path in $C_{free}$, each segment $L(v_k, v_{k+1})$ must be entirely contained in $C_{free}$.

% \subsubsection{Initial Tether Path and Shortcutting as a Homotopy}
% Let $\mathbf{P}_{init}(t)$ be an initial PL path representing the tether before shortcutting optimization at time $t$. This path connects $p_A$ to $p_R(t)$ and is assumed to be a PL approximation of $\Gamma_{ROV}$ lying in $C_{free}$. It is a standard result in algebraic topology that $\Gamma_{ROV} \sim \mathbf{P}_{init}(t)$ if $\mathbf{P}_{init}(t)$ is a sufficiently fine approximation.

% The shortcutting mechanism iteratively refines a PL path. Let $\mathbf{P}_{k} = (v_0, \ldots, v_M)$ be the PL tether path at iteration $k$. A shortcutting step selects two non-consecutive vertices $v_i, v_j \in \mathbf{P}_{k}$ (where $i < j-1$). Let $\sigma_{old}$ be the subpath of $\mathbf{P}_{k}$ from $v_i$ to $v_j$, and let $\sigma_{new}$ be the straight-line segment $L(v_i, v_j)$. If $\sigma_{new} \subset C_{free}$ (i.e., the line of sight is collision-free), the path $\mathbf{P}_{k+1}$ is formed by replacing $\sigma_{old}$ with $\sigma_{new}$.

% \textbf{Claim:} If $\sigma_{new} \subset C_{free}$, then $\sigma_{old} \sim \sigma_{new}$ in $C_{free}$ relative to their common endpoints $v_i$ and $v_j$.

% \textit{Proof of Claim:}
% The tether model is "analogous to a rope tightening around an obstacle." When a flexible rope segment, initially following $\sigma_{old} \subset C_{free}$, is pulled taut between $v_i$ and $v_j$, and the direct path $\sigma_{new}$ is also in $C_{free}$, the rope physically deforms into $\sigma_{new}$ by moving exclusively through $C_{free}$.
% This continuous deformation can be represented by a homotopy. Let $\sigma_{old}(s)$ parameterize the PL path $\sigma_{old}$ for $s \in [0,1]$, and $\sigma_{new}(s) = (1-s)v_i + s v_j$ parameterize the segment $\sigma_{new}$. A straight-line homotopy is $H(s, \tau) = (1-\tau)\sigma_{old}(s) + \tau\sigma_{new}(s)$ for $s, \tau \in [0,1]$.
% The "rope tightening" analogy implies that if $\sigma_{new} \subset C_{free}$, then the entire region swept by the deformation from $\sigma_{old}$ to $\sigma_{new}$ is also contained in $C_{free}$. Thus, $H(s, \tau) \in C_{free}$ for all $s, \tau$. This ensures that $\sigma_{old} \sim \sigma_{new}$ in $C_{free}$.
% Since $\mathbf{P}_{k+1}$ is obtained from $\mathbf{P}_{k}$ by substituting $\sigma_{old}$ with a homotopic segment $\sigma_{new}$, it follows that $\mathbf{P}_{k} \sim \mathbf{P}_{k+1}$.
% \hfill $\qed$

% \subsubsection{Iterative Application and Final Equivalence}
% The shortcutting algorithm generates a sequence of PL paths $\mathbf{P}_{init}(t) = \mathbf{P}_0, \mathbf{P}_1, \ldots, \mathbf{P}_{final}$, where $\mathbf{P}_{final}$ is the resulting taut-tether path, denoted $\mathbf{P}^{tether}(t)$. Since each step is a homotopy, $\mathbf{P}_0 \sim \mathbf{P}_1 \sim \ldots \sim \mathbf{P}_{final}$. By the transitivity of the homotopy relation, $\mathbf{P}_{init}(t) \sim \mathbf{P}^{tether}(t)$.

% Given that $\Gamma_{ROV} \sim \mathbf{P}_{init}(t)$, and $\mathbf{P}_{init}(t) \sim \mathbf{P}^{tether}(t)$, we conclude by transitivity that:
% $$ \Gamma_{ROV} \sim \mathbf{P}^{tether}(t) $$
% Thus, the taut-tether path $\mathbf{P}^{tether}(t)$, resulting from the shortcutting operations, is homotopically equivalent to the ROV's actual trajectory $\Gamma_{ROV}$ within the free configuration space $C_{free}$. The shortcutting process simplifies the path to a tighter representative within its original homotopy class with respect to the obstacles.



%%%%%%%%%%%%%%%%%%%%%%%%%%%%%%%%%%%%%%%%%%
%%%%%%%%%%%%%%%%%%%%%%%%%%%%%%%%%%%%%%%%%%
%%%%%%%%%%%%%%%%%%%%%%%%%%%%%%%%%%%%%%%%%%


If the shortcut is obstructed or the endpoint node \( p_j(t) \) is in collision, the model applies a pulling operation. This simulates disentanglement by moving \( p_j(t) \) incrementally toward the tether endpoint \( p_{N+1}(t) \), ensuring the new configuration remains collision-free. This iterative process continues until no further shortcuts or pulling is possible, resulting in an updated, taut, and obstacle-free tether path \( \mathbf{P}^{tether}(t+1) \). An example of the shortcut operation is illustrated in Fig.~\ref{fig:tether}.


% %%%%%%%%%%%%%%%%%%%%%%%%%%%
% %%%%% REACT Planner %%%%%%
% %%%%%%%%%%%%%%%%%%%%%%%%%%%
\begin{algorithm}[H]
\LinesNotNumbered  % Disable line numbers

\SetKwInOut{Input}{Input}
\SetKwInOut{Output}{Return}
\Input{
Waypoints $\mathbf{W}$, 
Maximum tether length $L_{\max}$, 
Current tether configuration $\mathbf{P}^{tether}(t)$, 
Current ROV position $\mathbf{p}_{\text{rov}}(t)$, 
Current waypoint index $k$, $d_{thresh}$}

\Output{Target point for controller $\mathbf{p}_{\text{target}}$}
\BlankLine

Compute $L_{\text{tether}}(t)$ 

\If{$L_{\text{tether}}(t) > L_{\max}$ \textbf{and not} finding\_safe\_path}{
    finding\_safe\_path $\gets$ True\; \tcp{Activate recovery mode}
    $\mathbf{P}_{\text{recovery}} \gets \text{SearchAlternativePath}(\mathbf{P}^{tether}(t), \mathbf{W}[k], L_{\max})$\; \tcp{Find path within tether limits}
    $\mathbf{P}_{\text{safe}} \gets \text{RefineRecoveryPath}(\mathbf{P}_{\text{recovery}})$\; %\tcp{Optimize the recovery path}
    path\_is\_safe $\gets \text{CheckPathValidity}(\mathbf{P}_{\text{safe}})$\; \tcp{Check if recovery path is valid}
    safe\_path\_index $\gets 0$\; \tcp{Start from the beginning of the recovery path}
}

\If{finding\_safe\_path \textbf{and} path\_is\_safe}{
    $\mathbf{p}_{\text{target}} \gets \text{GetPointAlongPath}(\mathbf{P}_{\text{safe}}, \text{safe\_path\_index})$\; \tcp{Get target from recovery path}
    \If{$\|\mathbf{p}_{\text{rov}}(t) - \mathbf{p}_{\text{target}}\| < d_{thresh}$}{
        safe\_path\_index $\gets$ safe\_path\_index + 1\; \tcp{Advance to next point}
    }
    \If{safe\_path\_index $\ge$ $|\mathbf{P}_{\text{safe}}|$}{
        finding\_safe\_path $\gets$ False\; \tcp{Recovery complete}
        $\mathbf{p}_{\text{target}} \gets \mathbf{W}[k]$\; \tcp{Resume wp tracking}
    }
}
\Else{
    finding\_safe\_path $\gets$ False\; \tcp{Reset recovery}
    $\mathbf{p}_{\text{target}} \gets \mathbf{W}[k]$\; \tcp{Default to current wp}

    \If{$\|\mathbf{p}_{\text{rov}}(t) - \mathbf{p}_{\text{target}}\| < d_{thresh}$}{
         $k \gets k + 1$\; \tcp{Advance to next wp}
    }
}
\Return{$\mathbf{p}_{\text{target}}$}\;
\caption{Real-time Entanglement-Aware Path Planning}
\label{alg:main_loop}
\end{algorithm}


%%%%%%%%%%%%%%%%%%%%%%%%%%%%%%%%%
%% Alg. 3 Recovery Path  Algorithm %%%
%%%%%%%%%%%%%%%%%%%%%%%%%%%%%%%%%

\begin{algorithm}[t]
\LinesNotNumbered  % Disable line numbers
\SetKwInOut{Input}{Require}
\SetKwInOut{Output}{Return}
\Input{
Current tether $\mathbf{P}(t)$,
Goal waypoint $\mathbf{p}_{\text{goal}}$,
Maximum length $L_{\max}$
}
\Output{Recovery path segment $\mathbf{P}_{\text{recovery}}$}
\BlankLine
best\_path $\gets$ None\;
prev\_len $\gets \infty$\;
found\_suitable $\gets$ False\;
\For{$i \gets |\mathbf{P}(t)| - 3$ \textbf{downto} 3}{
    $\mathbf{P}_{\text{candidate}} \gets \text{GeneratePath}(i, \mathbf{P}(t), \mathbf{p}_{\text{goal}})$\;
    $L_{\text{candidate}} \gets \text{ComputeLength}(i, \mathbf{P}(t), \mathbf{P}_{\text{candidate}})$\;\\
    \If{$L_{\text{candidate}} < 0.7L_{\max}$} %\textbf{and} $L_{\text{candidate}} < 0.65 \cdot \text{prev\_len}$}
    {
         best\_path $\gets \text{ExtractSegment}(i+2, \mathbf{P}(t))$\;
         found\_suitable $\gets$ True\;
         \textbf{break}\;
    }
    prev\_len $\gets \text{ComputeLength}(i-3, \mathbf{P}(t), \text{GeneratePath}(i-3, \mathbf{P}(t), \mathbf{p}_{\text{goal}}))$\;
}
\If{\textbf{not} found\_suitable}{
    $\mathbf{p}_{\text{rov}} \gets \mathbf{P}(t)[|\mathbf{P}(t)|-1]$\; \tcp{Fallback}
    $\mathbf{P}_{\text{recovery}}$ $\gets \text{PlanShortestPath}(\mathbf{p}_{\text{rov}}, \mathbf{p}_{\text{goal}})$\;
}
\Return{$\mathbf{P}_{\text{recovery}}$}\;
\caption{Search Alternative Recovery Path}
\label{alg:search_alternative}
\end{algorithm}

%%%%%%%%%%%%%%%%%%%%%%%%%%%%%%%%%%%%%%%%%%
%%%%%%%%%%%%%%%%%%%%%%%%%%%%%%%%%%%%%%%%%%
%%%%%%%%%%%%%%%%%%%%%%%%%%%%%%%%%%%%%%%%%%



%%%%%%%%%%%%%%%%%%%%%%%%%%%
%% Refine Path  Algorithm
%%%%%%%%%%%%%%%%%%%%%%%%%%%
\begin{algorithm}[b]
\LinesNotNumbered  % Disable line numbers
\SetKwInOut{Input}{Input}
\SetKwInOut{Output}{Return}
\Input{
    Recovery path $\mathbf{P}_{\text{recovery}}$;\\
    Offset distance $\delta_{\text{offset}}$;\\
    Sampling distance $\delta_{\text{sample}}$
}
\Output{Refined safe path $\mathbf{P}_{\text{safe}}$}
\BlankLine
$\mathbf{P}_{\text{offset}} \gets \text{TetherPathOffset}(\mathbf{P}_{\text{recovery}}, \delta_{\text{offset}})$\;
\\
$\mathbf{P}_{\text{sampled}} \gets \text{PerturbTetherPath}(\mathbf{P}_{\text{offset}}, \delta_{\text{sample}})$\;
\\
$\mathbf{P}_{\text{safe}} \gets \text{SmoothTetherPath}(\mathbf{P}_{\text{sampled}})$\;
\\
\Return{$\mathbf{P}_{\text{safe}}$}\;
\caption{Refine Recovery Path}
\label{alg:refine_path}
\end{algorithm}


%%%%%%%%%%%%%%
%%%%%%%%%%%%%%
%%%%%%%%%%%%%%

%%%%%%%%%%%%%%%%%%%%%%%%%%%
%%%%%%%%%%%%%%%%%%%%%%%%%%%
%%%%%%% PLANNER %%%%%%%%%
%%%%%%%%%%%%%%%%%%%%%%%%%%
%%%%%%%%%%%%%%%%%%%%%%%%%%





\section{Real-Time Entanglement-Aware Path Planner}
\label{sec:planner}

In this section, we present the real-time entanglement-aware path planner, a local planner designed to handle real-time entanglement-free path finding for autonomous system navigating operating with a tether. The planner runs in an online manner, continuously monitoring the tether configuration and adapting the \ac{ROV}'s path to avoid exceeding the maximum allowable tether length, \( L_{\text{max}} \), while ensuring safe, collision-free motion towards a sequence of reference waypoints.

At each time step \( t \), the planner utilizes the current estimated tether configuration  $\mathbf{P}^{tether}(t)$. The planner also maintains a list of reference waypoints \( \mathbf{W} = \{\mathbf{p}_{\text{waypoint}}(k)\}_{k=1}^{M} \), where \( \mathbf{p}_{\text{waypoint}}(k) \in \mathbb{R}^3 \) denotes the \( k \)-th target waypoint. The current tether length \( L_{\text{tether}}(t) \) is computed by summing the Euclidean distances between consecutive tether points, i.e., \( L_{\text{tether}}(t) = \sum_{i=1}^{N-1} \| p_i(t) - p_{i+1}(t) \| \), and is subsequently compared to the predefined maximum allowable length \( L_{\text{max}} \), to activate the re-planner.

\subsection{Nominal Operation}
If the tether length constraint is satisfied, i.e., \( L_{\text{tether}}(t) \leq L_{\text{max}} \), the planner operates in its nominal mode. It identifies the next reference waypoint \( \mathbf{p}_{\text{waypoint}}(k) \) from the list \( \mathbf{W} \) that has not yet been reached and directs the \ac{ROV} controller towards it. The system proceeds sequentially through the waypoints as long as the tether constraint remains satisfied.



\subsection{Entanglement Avoidance Strategy}
If the tether length exceeds the maximum allowable limit, \( L_{\text{tether}}(t) > L_{\text{max}} \), the entanglement avoidance strategy is activated. This strategy aims to guide the \ac{ROV} along a temporary recovery path to de-tangle the tether before resuming navigation towards the original target waypoint.

The planner initiates a search for a suitable recovery path by iteratively evaluating potential de-tangle paths based on the current tether configuration \( \mathbf{P}^{tether}(t) \). Starting from the \ac{ROV}'s end of the tether (node \( p_{N-1}(t) \)) and moving backward towards the anchor point \( p_1(t) \), the planner considers each node \( p_i(t) \) as a potential pivot point. For each \( i \), a candidate alternative trajectory is implicitly generated, consisting of the segment from the \ac{ROV} back to \( p_i(t) \) followed by a newly planned path segment from \( p_i(t) \) to the current target waypoint \( \mathbf{p}_{\text{waypoint}}(k) \). The planner computes the estimated length of this candidate alternative tether configuration.

A recovery path is selected based on length criteria: the planner seeks a pivot point \( p_i(t) \) such that the corresponding alternative path length is significantly shorter than \( L_{\text{max}} \) (e.g., \( < 0.7 L_{\text{max}} \)) and offers substantial improvement compared to alternatives generated using nearby pivot points (e.g., \( p_{i-3}(t) \)). Once such an index \( i \) is found, the recovery path segment \( \mathbf{P}_{\text{recovery}} \) is defined as the portion of the current tether from the \ac{ROV} back to a point slightly further back along the tether, specifically \( p_{i+2}(t) \). This segment represents the initial trajectory the \ac{ROV} must follow to begin resolving the entanglement. If no suitable recovery path is found during the backward search, a direct path from the \ac{ROV}'s current position to the goal waypoint is generated as a fallback. This recovery path search process is illustrated in Fig. \ref{fig:planner_search}.  










\subsection{Recovery Path Refinement and Execution}
The initially selected recovery path \( \mathbf{P}_{\text{recovery}} \) undergoes further refinement to enhance safety and smoothness before execution. This involves several steps:
\begin{enumerate}
    \item \textbf{Centroid Offsetting:} Points along \( \mathbf{P}_{\text{recovery}} \) are pushed slightly outwards, away from the path's geometric centroid, to increase clearance from potential obstacles near the path's center.
    \item \textbf{Random Sampling Perturbation:} Points are locally perturbed by sampling in random directions, seeking nearby collision-free states to potentially escape minor constraint violations or local minima.
    \item \textbf{Polynomial Smoothing:} A polynomial function (e.g., 3rd order) is fitted to segments of the path to generate a smoother trajectory, reducing sharp turns and improving dynamic feasibility.
\end{enumerate}
The resulting refined path is denoted as \( \mathbf{P}_{\text{safe}} \). The planner then checks if \( \mathbf{P}_{\text{safe}} \) is collision-free using the state validity checker. If the entanglement strategy is active and \( \mathbf{P}_{\text{safe}} \) is valid, the controller is directed to follow points sequentially along \( \mathbf{P}_{\text{safe}} \). Once the \ac{ROV} reaches the end of \( \mathbf{P}_{\text{safe}} \), the entanglement avoidance strategy is deactivated, and the planner reverts to nominal operation, targeting the next waypoint from the original list \( \mathbf{W} \). If \( \mathbf{P}_{\text{safe}} \) is found to be invalid (e.g., due to collisions introduced during refinement), the system continues targeting the original waypoint \( \mathbf{p}_{\text{waypoint}}(k) \), relying on lower-level collision avoidance or requiring further planning cycles. 

The core logic can be summarized in the following algorithms. Algorithm~\ref{alg:main_loop} outlines the main planning cycle, Algorithm~\ref{alg:search_alternative} details the search for the recovery path, and Algorithm~\ref{alg:refine_path} describes the path refinement process.














%%%%%%%%%%%%%%%%%%%%%%%%%%%%%%%%%%%%%%%%%%
%%%%%%%%%%%%%%%%%%%%%%%%%%%%%%%%%%%%%%%%%%
%%%%%%%%%%%%%%%%%%%%%%%%%%%%%%%%%%%%%%%%%%












%%%%%%%%%%%%%%%%%%%%%%%%%
%%%% Figure : Planner search 
%%%%%%%%%%%%%%%%%%%%%%%%
%%%%%%%%%%%%%%%%%%%%%%%%
\begin{figure*}[t!]
    \centering
    \includegraphics[width=\textwidth]{Phd_thesis/figures/planner.pdf}
    \caption{Recovery path search process: 
    (1) Following the path from node \( \mathbf{p}_3 \), denoted as \( \mathbf{P}_{r_3} \), causes the tether length \( L_{\text{tether}} \) to exceed the maximum allowed length \( L_{\text{max}} \); 
    (2) a backward recovery search is initiated from node \( \mathbf{p}_2 \) along the tether trajectory \( \mathbf{P}_{tether}(t) \), but the resulting path yields no improvement in tether length; 
    (3) continuing the search further back to node \( \mathbf{p}_1 \) leads to a feasible path and an updated, valid tether configuration; 
    (4) the final planned safe path \( \mathbf{P}_{safe} \) (green) satisfies the tether constraint (\( L_{\text{tether}} \leq L_{\text{max}} \)), ensuring an entanglement-free configuration.}
    \label{fig:planner_search}
\end{figure*}
%%%%%%%%%%%%%%%%
%%%%%%%%%%%%%%%%
 % 1.5 pages
\section{Experiments}
\label{sec:experiments}

% one point to discuss about teher length, extends operaion.
% dsicussion on how to define maximum tether length to entaglement.
% What is the common operational tether length in industry? 
%cost benefit analysis for tether saving vs time saving

% when comaping planner compare them relative to each other percentage

% Mention that after inspection, we need to deantgale so this will take time for CPP 

%Mention that this scaled model sceanrio 

%
%Offline vs online discussion
%

%online allows for arbitrary fixation.

% allows for suggestion for remotely 

% can be emebeded in offline planer

% Run one more shape for simulations.

% Add detangeling operaton.

This section presents the experimental results of our proposed \ac{REACT} method. The entire framework is implemented in C++ to ensure computational efficiency, and integrated with ROS to facilitate modularity and ease of deployment on real-world robotic systems. For the planning component, we utilize the \ac{OMPL} library \cite{ompl} to compute shortest paths using the RRT* algorithm.  


\subsection{Experimental Set Up}
The path planner is implemented for a BlueROV2 simulation model. A \ac{MPC} approach is chosen to account for model constraints and to provide the optimal control input $\mathbf{u}^{ref} \in \mathbb{R}^4$, where $\mathbf{u}^{ref} = [F_x, F_y, F_z, M_z]^T$ represents the forces in the $X$, $Y$, and $Z$ directions, and the moment about the $Z$-axis. The goal is to follow the desired reference trajectory $\mathbf{x}^{ref} \in \mathbb{R}^6$, where $\mathbf{x}^{ref} = [x_{ref}, y_{ref}, z_{ref}, \psi_{ref}]^T$ contains the reference position ($x_{ref}$, $y_{ref}$, $z_{ref}$) and the reference yaw angle $\psi_{ref}$.

The entanglement-aware path planner provides $\mathbf{x}^{ref}$ in real time, ensuring the BlueROV2 avoids obstacles while following the desired trajectory. For further details about the controller and model, the reader is referred to \cite{amergp}.

%\subsection{Results}
We perform a comparative analysis of the proposed \ac{REACT} method and a baseline conventional \ac{CPP} (FC-Planner \cite{feng2024fc}), which does not contain explicit entanglement handling. Both planners were tested in the same simulated pipe environment.

The simulation setup consists of a pipe structure represented by an underwater pipe model from \cite{feng2024fc}. The simulated onboard camera featured a horizontal field of view of 70 degrees and a vertical field of view of 60 degrees, with a maximum inspection range of 3.0 meters. The tether constraint was defined by a maximum allowable tether length of $L_{max}$ = 10.0 meters.

Environmental coverage was evaluated geometrically. At each subsampled time step, the position and orientation of the camera were used to determine which triangles in the environment mesh were visible. A triangle was marked as visible if its centroid was within the inspection range, its surface normal faced the camera, and its projection lay within the camera's field of view. Over time, the set of all uniquely observed triangles was accumulated. The coverage at time $t$ was defined as the ratio of the number of unique visible triangles up to time $t$ to the total number of triangles in the environment.


\subsection{Results}

The quantitative performance metrics for both planners are summarized in Table~\ref{tab:performance_metrics}.

\begin{table}[ht]
    \centering
    \caption{Performance Metrics Comparison}
    \label{tab:performance_metrics}
    \resizebox{\columnwidth}{!}{%
    \begin{tabular}{|l|c|c|c|}
        \hline
        \textbf{Planner} & \textbf{Total Time (s)} & \textbf{Tether Exceeded (s)} & \textbf{Final Coverage (\%)} \\
        \hline
        \ac{REACT} & 546.00 & 93.12 & 99.91 \\
        CPP  & 429.00 & 96.58 & 99.82 \\
        \hline
    \end{tabular}%
    }
    \vspace{0.5em}
\end{table}


Figures~\ref{fig:coverage_vs_time} and~\ref{fig:tether_vs_time} visualize the performance of the planners in terms of environmental coverage and tether length, respectively.

\begin{figure}[ht]
    \centering
    \begin{subfigure}[b]{0.48\linewidth}
        \centering
        \includegraphics[width=\linewidth]{figures/coverage_vs_time.pdf}
        \caption{Environmental Coverage vs. Time for both planners.}
        \label{fig:coverage_vs_time}
    \end{subfigure}
    \hfill
    \begin{subfigure}[b]{0.48\linewidth}
        \centering
        \includegraphics[width=\linewidth]{figures/tether_length_vs_time.pdf}
        \caption{Tether Length vs. Time, highlighting the 10.0m maximum limit.}
        \label{fig:tether_vs_time}
    \end{subfigure}
    \caption{Comparison of coverage and tether behavior over time.}
    \label{fig:coverage_tether_sidebyside}
\end{figure}


\begin{figure}[ht]
    \centering
    \begin{subfigure}[b]{0.48\linewidth}
        \centering
        \includegraphics[width=\linewidth]{figures/fc_planner_final_view.pdf}
        \caption{3-D plot CPP.}
        \label{fig:3d_cpp}
    \end{subfigure}
    \hfill
    \begin{subfigure}[b]{0.48\linewidth}
        \centering
        \includegraphics[width=\linewidth]{figures/ta_planner_final_view.pdf}
        \caption{3-D plot, \ac{REACT}.}
        \label{fig:3d_oea}
    \end{subfigure}
    \caption{3D views of final trajectories. (a) CPP results in entangled tether geometry. (b) \ac{REACT} yields a non-entangled tether configuration, reflecting effective entanglement avoidance.}
    \label{fig:3Dplots}
\end{figure}




The results highlight distinct trade-offs between the two planning strategies. The \ac{CPP}, lacking entanglement awareness, completed the trajectory significantly faster (429.00~s vs. 546.00~s) while achieving marginally less final coverage (99.81\% vs. 99.91\%).

However, focusing solely on speed and final coverage overlooks the critical aspect of tether management in constrained environments. As shown in Figure~\ref{fig:tether_vs_time}, both planners exceeded the $L_{\text{max}}$ constraint during execution. The \ac{CPP} exceeded the limit at 96.58~s, slightly later than the Online Entanglement-Aware PP at 93.12~s. 

This metric requires careful interpretation. \ac{REACT} is explicitly designed to avoid and react to potential tether entanglement. Its reactive replanning often involves maneuvers specifically intended to reposition the \ac{ROV} and adjust the tether configuration for safety. These maneuvers can temporarily increase tether length, potentially causing earlier—but controlled and intentional—exceedances of the simple length threshold as the planner actively works to prevent more critical entanglement states later in the mission.

Conversely, the \ac{CPP}, lacking this foresight and reactivity, proceeds along its path until physical constraints force a violation. While this happened slightly later in this simulation, the approach risks encountering more severe or unrecoverable tether states without the ability to proactively mitigate them.

The longer execution time of the \ac{REACT} directly reflects the computational cost of tether modeling, collision checking for the tether, and executing reactive replanning maneuvers when necessary. This deliberate approach prioritizes tether safety and mission robustness over raw speed, offering a significant advantage for reliable operation in complex, real-world underwater scenarios where tether integrity is paramount. 

The \ac{CPP}'s speed advantage comes at the cost of ignoring potential tether hazards, making it less suitable for missions where entanglement poses a significant risk. Therefore, \ac{REACT} demonstrates superior performance in the context of safe and robust tethered \ac{ROV} operation, despite the longer completion time observed in this comparison.












 % 

\section{Conclusion & Future Work}
\label{sec:conclusion}
We presented \ac{REACT} a real-time entanglement-aware path planning framework for tethered \ac{ROV}'s operating in constrained underwater environments. Through comparative simulations, we demonstrated that while conventional planners may complete missions faster, they lack the ability to handle tether-related challenges effectively. \ac{REACT} mitigates potential entanglement through reactive replanning, offering a more robust and reliable solution. Despite a moderate increase in execution time, the entanglement-aware planner enhances mission safety and integrity—making it a preferable choice for high-risk, cluttered inspection scenarios where tether management is critical. 

As future work, an experimental validation setup will be established to evaluate \ac{REACT} in a controlled underwater environment. 



%Furture work will involve creating a ros plugin and integration with the UNav-sim simulator.


 % 0.25 page

%\appendix
%\section{Projection and Back-Projection}
%\input{sections/projection}

\section*{Acknowledgment}
This research receives support from Innovation Fund Denmark grant number 2040-00032B and EIVA a/s. The authors would also like to express their gratitude to Dr. Louis Petit from Université de Sherbrooke for his valuable insights on the ropeRRT path planner method and for open-sourcing his implementation using the OMPL library.
\bibliographystyle{IEEEtran}
% argument is your BibTeX string definitions and bibliography database(s)
\bibliography{References.bib}
%
% <OR> manually copy in the resultant .bbl file
% set second argument of \begin to the number of references
% (used to reserve space for the reference number labels box)

\end{document}
