\label{sec:analysis_results}

The findings summarized above provide strong evidence validating the central hypothesis of this thesis. The demonstrated results show that enhancing \ac{MPC} with synergistic advancements in data-driven modeling, entanglement-aware  planning , and perception-aware control indeed leads to superior performance in complex marine robotic tasks.

The research effectively addressed the key challenges initially identified. The simulation-reality gap was narrowed by the development of UNav-Sim, facilitating more trustworthy validation. The detrimental effects of environmental uncertainty and model mismatch on \ac{MPC} performance were significantly mitigated through the adaptive learning capabilities of DF-GP and the improved predictive power of PINC, resulting in enhanced robustness and tracking precision. The critical mission-safety tradeoffs, particularly tether entanglement for \acp{ROV}, were directly managed by the REACT framework, enabling safe operation in previously inaccessible or hazardous scenarios. Finally, the perception-control decoupling issue was tackled by the VT-NMPC approach, demonstrating how task objectives can be integrated into the controller for improved mission effectiveness.

The synergy between these advancements is crucial. More accurate and adaptive models derived from data-driven techniques allow the \ac{MPC} to operate more effectively. Safety-aware planners like REACT ensure this enhanced control respects operational constraints, while perception-aware formulations guide the robot to maximize task success. Validating these integrated systems is made feasible through realistic simulation. Therefore, the collective evidence confirms that this multi-faceted approach, enhancing \ac{MPC} from different angles, yields autonomous systems that are safer, more robust, and more capable than those based on standard techniques, particularly for demanding underwater inspection missions.

