\chapter{Entaglement-Aware Coverage Path Planning}
\label{ch:react}

Autonomous inspection of marine structures is one of the most common and critical applications of underwater robotics. These structures are typically deployed in harsh environments for extended periods, requiring frequent and reliable monitoring to ensure their integrity and performance. Traditional inspection methods using human divers are often costly, risky, and logistically challenging. In this context, coverage becomes a key requirement for effective inspection. While numerous coverage path planners (CPP) have been proposed in the literature, a critical limitation remains largely unaddressed: the risk of tether entanglement during inspection with tethered vehicles.
 
 In this chapter we present REACT, a novel real-time coverage path planning framework for tethered underwater vehicles, designed to enable safe and efficient inspection of underwater structures. Unlike traditional CPP frameworks that does not take into account risk entanglement, REACT integrates a fast geometry-based tether model utilizing signed distance fields (SDFs) to simulate the tether configuration in real time. By enforcing a maximum tether length constraint, REACT proactively prevents entanglement during mission execution.

The REACT framework operates by coupling this tether model with a MPC and an off-the-shelf CPP, facilitating online replanning and optimal trajectory tracking. The system has been validated in simulated pipe inspection tasks, where it demonstrated the ability to maintain full coverage while avoiding tether-related failures.

We also provide a comprehensive review of prior work in tether-aware planning, highlighting key limitations such as assumptions of 2D environments, reliance on offline computation, and lack of generalization to 3D scenarios. In contrast, REACT addresses these challenges by offering a scalable and adaptable solution suitable for complex and cluttered 3D underwater environments.

The chapter concludes with simulation results showing that, although REACT may incur a slight increase in path execution time, it ensures mission robustness by eliminating the risk of tether entanglement—making it a superior choice for real-world underwater inspection tasks where safety and reliability are paramount. 