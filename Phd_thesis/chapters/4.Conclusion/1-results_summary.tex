\chapter{Conclusion \& Future Work}
This thesis confronted significant obstacles impeding robust autonomy in marine robotics. The core objective was to substantially enhance the performance and reliability of \ac{MPC}-based systems by integrating advancements across simulation, control, planning, and data-driven modeling. The underlying hypothesis guiding this work proposed that such an integrated approach would empower autonomous marine robots to execute complex missions, like underwater inspections, with improved safety, efficiency, and adaptability compared to conventional methods. Through the development and validation of several novel frameworks and methodologies, this research has made progress towards realizing this vision.

\section{Results Summary}


\label{sec:summary_contributions}

The research presented in this thesis yielded significant advancements in autonomous marine system capabilities. A foundational achievement was the development of \textbf{UNav-Sim}, a high-fidelity, open-source simulation environment leveraging UE5. This tool provides the necessary visual and physical realism to effectively bridge the simulation-to-reality gap, enabling more reliable development and pre-deployment testing of complex perception and control algorithms.

Building upon this foundation, crucial progress was made in enabling safer and more effective navigation, particularly for challenging inspection tasks. For tethered \acp{ROV} operating in complex, constrained environments, the \textbf{REACT} framework demonstrated the ability to perform comprehensive coverage planning while actively preventing tether entanglement. By employing real-time, geometry-aware tether modeling and reactive replanning, safe navigation was achieved in scenarios previously considered high-risk, ensuring mission integrity. Furthermore, the developement of perception-aware objectives, through the VT-NMPC method, showed how MPC can directly optimize for task-specific objectives, such as maintaining optimal visual inspection viewpoints, rather than merely following a predefined path. This approach promises more robust and higher-quality inspection.

Complementing these planning and control advancements, the thesis demonstrated methods to significantly enhance controller robustness against the uncertainties inherent in the marine environment. By incorporating physics-informed machine learning techniques (\textbf{PINC}), more accurate and generalizable dynamic models of underwater vehicles were developed, leading to improved long-horizon prediction capabilities crucial for \ac{MPC}. Moreover, the development of adaptive online learning methods (\textbf{DF-GP}) enabled \ac{MPC} frameworks to effectively estimate and compensate for time-varying environmental disturbances in real-time, drastically improving trajectory tracking accuracy  without requiring manual retuning for different conditions. Collectively, these results represent a substantial step towards more intelligent, adaptable, and reliable autonomous marine robots.