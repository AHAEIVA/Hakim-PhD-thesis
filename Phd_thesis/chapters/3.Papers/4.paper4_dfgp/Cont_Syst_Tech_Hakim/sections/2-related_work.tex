\section{related work}
\label{sec:related_work}

Model learning is a critical component of robot control, enabling precise and robust model-based controller design. One approach focuses on addressing errors in individual parameters separately \cite{aero, madebo2024robust, tiltrotor}, while another consolidates all unmodeled effects into a single lumped disturbance term.
%Two fundamental techniques exist for model learning: 
%Addressing errors in each parameter individually \cite{  aero, madebo2024robust} and consolidating all unmodeled effects into a lumped unknown disturbance term.
%\end{enumerate}
%\item  Addressing errors in each parameter individually \cite{package, 3dprint, aero, tractor, tiltrotor,madebo2024robust}.
In this work, we adopt the latter method as it renders a more generic model description. Multiple strategies exist for lumped disturbance estimation. 

One such approach underlies characterizing the unknown disturbances by a parametric model, for instance, a \ac{NN} \cite{salzmann2023real, neural-fly}. However, large \ac{NN} models suffer from issues including insufficient excitation, estimation windup, and the necessity for a large amount of training data. Alternative methods for learning disturbances involve utilizing a disturbance observer, such as an extended Kalman filter \cite{rov_ekf2}, or adopting stochastic models like \ac{GP} \cite{bionic_gp}.
% \cite{ekf_eth, rov_ekf1, rov_ekf2}

Unlike other models, \ac{GP} is a non-parametric regression technique that does not impose structure-specific requirements on the learned nonlinear models. Instead, it relies on general assumptions regarding continuity and differentiability. Moreover, a \ac{GP} model provides uncertainty quantification, enabling the implementation of conservative, thus safer control strategies \cite{cautious_mpc, prajapat2024safe}. \ac{GP} has also proven real-time applicability, as shown in \cite{real-gp-mpc, wono1}. In \cite{mohit_gp}, a cascaded GP architecture accurately learns wind disturbances on a drone based on a historical sequence of disturbance observations. A hierarchical control scheme, as presented in \cite{cao2017gaussian}, uses constrained \ac{GP}-based \ac{MPC} for translational and rotational subsystems, effectively simplifying nonlinear \ac{MPC} problems into convex ones. Additionally, \cite{maiworm2021online} presents a framework that integrates output feedback \ac{MPC} with evolving \ac{GP}, ensuring constraint satisfaction and stability while reducing computational load. Within high-speed indoor flight, \cite{data_driven_mpc} utilized a \ac{GP} model to capture unmodeled complex drag effects. \cite{wono2}  integrated a \ac{MHE} with a \ac{GP} to achieve simultaneous state estimation and disturbance rejection. 

Although the methods mentioned above have demonstrated significant enhancements over conventional \ac{MPC} algorithms and have shown practical viability in real-time applications, a notable challenge they encounter is their adaptability to unforeseen conditions. This challenge is associated with \ac{GP} models that rely on pre/offline tuned hyperparameters, facilitating real-time application but potentially limiting their ability to respond to dynamic environments. Recently, a fast online adaptive \ac{GP} incorporating a forgetting factor into the  \ac{GP} formulation has been proposed \cite{asgp}. In essence, it facilitates the adaptation of time series models that exhibit rapid variations. While this approach has shown promising results, selecting the appropriate forgetting factor online remains unresolved.

Hence, this work builds on \cite{mohit_gp} and \cite{asgp}, introducing a partitioned dataset to explicitly capture mean and transient disturbance components, while employing a forgetting factor to balance their contribution to disturbance prediction. Inspired by locally weighted models \ac{GP} \cite{local_gp}, which weighs \ac{GP} models with different training data and hyperparameters, a weighted average is applied to \ac{GP} models with shared data and hyperparameters but differing forgetting factors. The weight of each \ac{GP}  is then optimized over a horizon, enabling adaptation to disturbances without the need for an online hyperparameter optimization.

%We also propose dynamic \ac{GP} optimization with adaptive weighting, eliminating manual hyperparameter tuning. The forgetting factor $\lambda$ is optimized via a simple real-time linear program, enhancing adaptability and efficiency in disturbance handling.}

%Hence, in this work, we propose a novel framework for disturbance estimation, where the contributions of different \ac{GP} models with varying forgetting factors are optimized, creating a versatile \ac{GP} capable of adapting to diverse operating conditions.


%In this manuscript, we consider the latter method, since it allows for a more general model description. The unknown disturbances can be characterized by a parametric model, such as a neural network \cite{neural_mpc, neural-fly}. However, this approach comes with its limitations, such as insufficient excitation, estimation windup, and the necessity for a large amount of training data. Alternative methods for learning disturbances involve the utilization of a disturbance observer, such as employing an extended Kalman filter \cite{ekf_eth, rov_ekf1, rov_ekf2}, and stochastic models like using \ac{GP} \cite{jung2022gaussian, bionic_gp}. 

%Unlike other models, \ac{GP} does not impose specific structures on the learned nonlinear models. Instead, they rely on general assumptions regarding continuity and differentiability. Moreover, \ac{GP} provides uncertainty quantification, enabling the implementation of more conservative and consequently safer control strategies \cite{cautious_mpc, kraus_safe,polymenakos2020safety, prajapat2024safe}. \ac{GP} has also proven real-time applicability, as shown in \cite{real-gp-mpc, wono1}. In \cite{mohit}, wind disturbances acting on a drone are learned based on a historical sequence of disturbance observations using a cascaded \ac{GP} architecture. Within high-speed indoor flight, \cite{data_driven_mpc} utilized a \ac{GP} to capture unmodeled complex drag effects. Similarly, addressing the challenges of high-speed flight, \cite{wono2} integrates a moving horizon estimator with a \ac{GP} to achieve simultaneous state estimation and disturbance rejection. Although the methods mentioned above have demonstrated significant enhancements over conventional \ac{MPC} frameworks and have shown practical viability in real-time applications, a notable challenge they encounter is their adaptability to unforeseen conditions. This challenge arises because \ac{GP} models often rely on hyperparameters that are tuned offline, potentially limiting their ability to effectively respond to dynamic environments.

%Recently, a fast online adaptive \ac{GP} incorporating a forgetting factor into the \ac{GP} formulation is proposed in \cite{asgp}. This facilitates the adaptation of time series models that exhibit rapid variations. While this approach has shown promise, particularly for disturbance estimation in \ac{AUV}s given the dynamic nature of disturbances encountered during operation, the issue of selecting the appropriate forgetting factor online remains unresolved. Hence, a novel framework is proposed in this work for disturbance estimation, where the contributions of different \acp{GP} with varying forgetting factors are optimized, creating a versatile \ac{GP} capable of adapting to diverse operating conditions.







%In cases where a nominal model is unavailable, the research explores the use of fully data-driven models for nonlinear Model Predictive Control (MPC). Various approaches from the literature, such as autoregressive exogenous (ARX) models \cite{arx}, dynamic mode decomposition (DMD) \cite{dmd}, sparse identification of nonlinear dynamics (SINDy) \cite{sindy}, neural networks \cite{NN}, and Hankel matrix representation \cite{DD-MPC}, are considered for learning the full dynamical model from previously collected trajectories. 
%Other more efficient approaches to model learning involves utilizing a nominal model, $f(\mathbf{x},\mathbf{u})$, and then learning the model errors, $\mathbf{d}$. This method is more data-efficient as it incorporates prior information about the structure of the state transition function. Gaussian processes (GPs) have shown promise in learning unknown latent functions from data. GPs, being non-parametric, do not impose a specific structure on models and adapt quickly to changing data without overfitting. Furthermore, GPs provide a measure of uncertainty for predictions, allowing integration into a safe learning framework. Transfer learning capabilities of GPs are investigated, where model disturbances are learned from data collected during various tasks performed by the robot, inspired by the work in \cite{multitask,local_go}.


%As an alternative, some works have attempted to estimate disturbances using onboard sensors. For example, \cite{gust_indi} employed incremental non-linear dynamic inversion (INDI) to accurately control the position of a drone subject to wind gusts, using accelerometer data to calculate the required motor speeds to counteract the disturbances. Other related works have focused on combining model predictive control (MPC), which allows for prediction of future costs, with an estimator of unknown stochastic disturbances. This approach typically involves modeling the system dynamics as comprising a nominal part, modeled to high accuracy using first principles, and a stochastic part, which is estimated using various methods. \cite{ekf_eth} and \cite{nmhe} used an extended Kalman filter and a nonlinear moving horizon estimator, respectively, to estimate the disturbances acting on a drone. Other approaches have utilized neural networks to estimate the unknown dynamics of the model, such as \cite{neuro_bem}, which used a neural network to estimate nonlinear aerodynamic effects on a high-speed drone, and \cite{neural_mpc}, which employed a neural network in conjunction with MPC. Gaussian process regression is also a popular method for estimating disturbances, as it can provide a measure of uncertainty in the estimate and has been widely used in conjunction with MPC by the robotics community. \cite{mohit_gp} and \cite{data_mpc_eth} used GP to directly estimate the disturbance term for a UAV, and \cite{mpc_arm} combined GP with an extended Kalman filter to estimate disturbances on a robotic arm.}
\label{sec:problem_formulation}

