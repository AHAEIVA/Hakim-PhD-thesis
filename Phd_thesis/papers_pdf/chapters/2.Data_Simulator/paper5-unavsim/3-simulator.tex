\section{UNav-Sim software architecture} \label{sec:description}


UNav-Sim is composed of three main components, as illustrated in Fig. \ref{fig:abstract}: an underwater physics simulator, a state-of-the-art rendering engine, i.e. \ac{UE5}, and an autonomy stack. The underwater physics simulator, which contains the lumped parameter \ac{ROV} model and underwater dynamics equations, is modular and allows underwater vehicle motion simulation. %, UNav-Sim is implemented as a plugin that can be added to any \ac{UE5} project.  
It leverages the capabilities of AirSim, including the Fastphysics solver and a range of sensor models, such as GPS, IMU, cameras, and distance sensor. %Additionally, the modularity of AirSim allows for the seamless integration of custom underwater vehicle models into the simulator. 
An API allows communication between the navigation stack and the physics simulator, with the former receiving essential sensor data and sending control commands. A \ac{ROS} wrapper is also available, which enables \ac{ROS}-based development and communication between different modules.


%Our simulator, similar to AirSim, focuses on a modular and extensible design. The components of the simulator include an environment and vehicle model, a physics engine, sensor models, a rendering interface, a public API layer, and an interface layer for vehicle firmware, as illustrated in Figure \ref{fig:abstract}. During the simulation process, sensor data from the simulated environment is supplied to the navigation stack, which then generates actuator signals as input for the vehicle model component of the simulator. The purpose of the vehicle model is to compute the forces and torques generated by the simulated actuators. \ac{UWRS} utilizes existing sensors in AirSim, which includes, GPS, distance sensor (sonar), RGBD cameras, magnetometer and barometer. These sensor models are equipped with the capability to simulate sensor noise and degradation, which closely emulate real-world conditions. Furthermore, the simulator also enables the incorporation of custom sensors and sensor fusion techniques. 




\subsection{Underwater environment rendering}





Underwater image formation can be modelled as a superposition of absorption, forward scattering, and backscattering effects at each pixel $\textbf{x} = (u, v)$. The image intensity $I_c(x)$ in each color channel $c$ can be expressed as \cite{alvarez2019generation}:

\begin{equation}\label{eq:lighting_model}
I_c(x) = D_c(x) + F_c(x) + B_c(x)
\end{equation}

In this equation, $D_c$ represents the attenuated signal from the object due to absorption. The forward scattering component $F_c$ captures the light from the object that reaches the camera with small-angle scattering. Lastly, the backscattering component $B_c$ accounts for the degradation in color and contrast caused by the water scattering effect, where the light does not originate directly from the object. These effects can be modelled using different techniques and can vary based on the implementation within the rendering engine.









UNav-Sim utilizes \ac{UE5} as the rendering engine, which offers significant improvements over its predecessor, \ac{UE4}. \ac{UE5} significantly boosts polygon handling to 10 billion, introduces real-time ray-based lighting with Lumen, and incorporates Temporal Super Resolution for high-quality textures with minimal performance impact, enhancing visual fidelity and efficiency. 

%by supporting real-time ray tracing, allowing the creation of more detailed rendering. It also includes better collaboration tools and better hardware performance, thus making it a state-of-the-art rendering tool.

% the priority here is to correlate the equation and the terms earlier with the things you mention here....
\ac{UE5} underwater rendering module models scattering effects in underwater images \eqref{eq:lighting_model}, with Schlick Phase Functions \cite{schlick}, taking into account the Opaque or Masked water surface. 
The transparency of the water is implicitly handled within the volume shading model, % ...so maybe say here the "underwater imaging formation is implicitly [...], so that the relationship with the equation is clearer?
and refraction is managed by reading the depth and color beneath the water surface to distort the samples.  
One of the main challenges in generating underwater renderings is the variety of imaging conditions that drastically change the environment's appearance \cite{akkaynak2017space}. \ac{UE5} allows users to define the scattering coefficients, absorption coefficients, phase function, and color scale behind the water, providing control over the water's appearance and thus allowing users to simulate their preferred environment's conditions. 
%\ac{UE5} renders realistic water images by utilizing a combination of advanced material shaders, real-time reflections and refractions, wave simulation, surface foam and splashes, lighting and global illumination techniques, and post-processing effects. 


Consequently, using \ac{UE5} within UNav-Sim, underwater environments that appear realistic can be created, where \ac{UE5} allows designers to place and manipulate assets in a 3D space. 
These assets can include terrain, static meshes, and lighting. They can be customized to create underwater virtual worlds, as shown in Fig. \ref{fig:abstract}. 

UE uses blueprints to define the physical representation and behaviour of an \ac{ROV}. In UNav-Sim, the blueprint is linked to an external underwater physics engine to obtain kinematic information. The blueprint also defines cameras that gather visual information from the underwater environment, such as RGB and depth images. 



\subsection{Underwater physics}

The core of the physics underlying underwater vehicles consists of the equations of motion that describe the different forces and moments acting on the vehicle's body. These forces and moments can be classified into three categories: hydrostatics, hydrodynamics, and externally applied forces.

% Hydrostatic forces, which are represented by the function $g(\eta)$, are independent of the vehicle's velocity and arise from gravity and buoyancy, along with associated moments and torques. Hydrodynamic forces, on the other hand, are velocity-dependent and include added mass effects, Coriolis forces, and drag forces. These forces arise from the interaction between the vehicle and the surrounding water and can significantly impact the vehicle's behavior.
% External forces, denoted as $\tau$, include the forces exerted on the \ac{ROV} by its thrusters, as well as the disturbances, caused by the surrounding water flow. Multiple disturbance models, such as constant value, sinusoidal, and a combination of sine waves are implemented.


%To account for these forces, multiple current models, such as constant value, sinusoidal, and a combination of sine waves are implemented.

The equation of motion in the body-fixed frame, originally presented in Fossen \cite{fossen}, can be expressed in SNAME notation \cite{SNAME} as, 
%
\begin{equation}\label{eq:eom}
  \begin{multlined}
    \mathbf{M}_{RB}\dot{\mathbf{\nu}}= \tau - \underbrace{ \mathbf{C}_{RB}(\mathbf{\nu})\mathbf{\nu} }_\text{Coriolis \hspace{1 pt} term} - \underbrace{ \mathbf{M}_A \dot{\mathbf{\nu}} - 
    \mathbf{C}_A(\mathbf{\nu}) \mathbf{\nu} }_\text{Added\hspace{2pt}mass}     
    \\
    - \underbrace{ \mathbf{D}(\mathbf{\nu})\mathbf{{\mathbf{\nu}} }}_\text{Drag } - \mathbf{g}(\mathbf{\eta}).
    \end{multlined}
\end{equation}
%
The vehicle's pose, $\mathbf{\eta}=[x, y, z, \phi, \theta, \psi]^T$, is described by a six-dimensional column vector where $x$, $y$, and $z$ denote the vehicle's position in the \ac{NED} frame, while $\phi$, $\theta$, and $\psi$ represent its roll, pitch, and yaw angles, respectively (see Fig. \ref{fig:rov_fbd}).
The linear and angular velocity vector in the body-fixed frame is denoted as $\mathbf{\nu} = [u, v, w, p, q, r]^T$. The inertia matrix of the vehicle's body is represented by $\mathbf{M}_{RB}$.
Hydrostatic forces, $g(\eta)$, arise from gravity and buoyancy, along with associated moments and torques.
Hydrodynamic forces arise from the interaction between the vehicle and the surrounding water and can significantly impact the vehicle's behavior. These forces include the Coriolis and centripetal forces caused by the rigid body's mass, $\mathbf{C}_{RB}(\mathbf{\nu})\mathbf{\nu}$, the Coriolis forces, $\mathbf{C}_A(\mathbf{\nu})$, and moment of inertia, $\mathbf{M}_{A}$, arising from the added mass, and linear and quadratic damping effects, $\mathbf{D}(\mathbf{\nu})\mathbf{{\nu} }$.
External forces, $\tau$, include the forces exerted on the \ac{ROV} by its thrusters, as well as the disturbances, caused by the surrounding water flow. Multiple disturbance models, such as constant value, sinusoidal, and a combination of sine waves are implemented.


To efficiently solve the equations of motion presented above, AirSim's high-frequency physics engine is utilized with a computational frequency of $1000 \textrm{Hz}$. The engine uses the velocity verlet algorithm for numerical integration due to its computational benefits.


%The algorithm is given by the following equation:
%\begin{equation}\label{eq:solver}
 %\tau_i = C_T \rho \omega^2_{max} D^4 u_{fi} 
 %\mathbf{v}_{k+1} = \mathbf{v}_{k} + \frac{ \dot{\mathbf{v}}_{k} + \dot{\mathbf{v}}_{k+1}}{2} \Delta t,
%\end{equation}
%where $\mathbf{v}_{k}$ and $\mathbf{v}_{k+1}$ are velocities in consequtive discrete timesteps, $k$ and $k+1$, and $\Delta t=10^{-3}\textrm{s}$ is the time step.

\begin{figure}[!t]
    \centering
    \includegraphics[width=.7\columnwidth]{figs/paper5-unavsim/bluerov_fbd.pdf}
    \caption[ROV model defined in UNav-Sim]{ROV model defined in UNav-Sim: A force $\tau_i$ is applied at each thruster location $i$, where the thruster orientation is defined by a vector $\textbf{n}_i$.}
    \label{fig:rov_fbd}
\end{figure}

\subsection{ROV model }


%The \ac{ROV} is represented as a rigid body that is manipulated by an arbitrary number of actuators. These actuators are located at user-defined vertices of the vehicle, with corresponding normals and positions denoted by $n_i$ and $r_i$, respectively, as shown in Fig. \ref{fig:rov_fbd}. At each vertex, a unitless control input, $u_i$ ($-1<u_i<1$) is used to regulate the rotational speed of the propellers, $\omega_i$. The resulting propeller thrust generates a force on the body vertices and propels the vehicle. A quadratic relationship is assumed between the propellers' thrust force and rotational speed. To accurately model the dynamics of the \ac{ROV}, several key parameters must be defined by the user, such as its inertia, center of buoyancy, hydrodynamic coefficients, and the maximum thrust force generated by each propeller. The physics engine uses these parameters to calculate the rigid body dynamics of the vehicle.


The \ac{ROV} is represented as a rigid body that is manipulated by an arbitrary number of actuators, $N$. These actuators are located at user-defined vertices of the vehicle, with corresponding normals and positions denoted by $n_i$ and $r_i$, respectively, where $i \in \{1,\dots, N\}$ is the actuator number. % as shown in Fig. \ref{fig:rov_fbd}. 
At each vertex, the vehicle receives a unitless control input, $u_i\in (-1,1)$. To account for actuator dynamics, a discrete low-pass filter with a time constant of $t_c$ is applied to the control input. The filtered input, denoted as $u_{fi}$, is then used to calculate the thrust force using the relationship given as \cite{rw:shah2018airsim},
%
\begin{equation}\label{eq:thruster_model}
 %\tau_i = C_T \rho \omega^2_{max} D^4 u_{fi} 
  \tau_i = C_T \rho \omega^2_{max} D^4 u_{fi}, 
\end{equation}
%
where $\tau_i$ is the thrust force on the $i$-th thruster, $C_T$ is the thrust coefficient, $\rho$ is the density of water, $\omega_{max}$ is the maximum thruster rotation speed, and $D$ is the propeller diameter. 
In order to accurately compute the specific rigid motion of the \ac{ROV}, as described by equations \eqref{eq:eom} and \eqref{eq:thruster_model}, several vehicle-specific parameters such as its inertia, hydrodynamic coefficients, and the maximum thruster rotation speed, must be configured by the user. %The model of Blue Robotics' BlueROV2 Heavy is implemented with $N=8$ thrusters along with UNav-Sim.

The model chosen to be implemented, the Blue Robotics BlueROV2 Heavy, is an over-actuated \ac{ROV} with four vertical thrusters and four horizontal thrusters as shown in Fig. \ref{fig:rov_fbd}. The horizontal thrusters are oriented at 45 degrees and are responsible for the control of three degrees of freedom, namely, surge, sway, and yaw, while the vertical thrusters control heave, pitch, and roll. The model parameters are obtained from \cite{bluerov2_h}. 




%\begin{equation}\label{eq:eom}
 %\tau_i = C_T \rho \omega^2_{max} D^4 u_i 
%  \tau_i = C_T \rho \omega^2_{max} D^4 u_i 
%\end{equation}


