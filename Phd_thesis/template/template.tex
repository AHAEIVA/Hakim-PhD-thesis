%%%%%%%%%%%%%%%%%%%%%%%%%%%%%%%%%%%%%%%%%
% AUthesis book
% LaTeX Template
% Version 1.0 (23/10/2023)
%
% This template is based on a template from:
% http://www.LaTeXTemplates.com
%
% Original author:
% Lorenzo Pantieri (http://www.lorenzopantieri.net) with extensive modifications by:
% Vel (vel@latextemplates.com), and
% Morten (morten.haastrup@eng.au.dk).
% Adaptation to Aarhus University and extension of the original template by:
% Mario J. Rincon (mjrp@mpe.au.dk)
%
% License:
% cC BY-NC-SA 3.0 (http://creativecommons.org/licenses/by-nc-sa/3.0/)
%
%%%%%%%%%%%%%%%%%%%%%%%%%%%%%%%%%%%%%%%%%

%----------------------------------------------------------------------------------------
%	REQUIRED PACKAGES
%----------------------------------------------------------------------------------------
\usepackage{xcolor}  % to define colours
\definecolor{AUpantone}{rgb}{0.0, 0.2392156, 0.5111111}
\definecolor{AUpantoneDark}{rgb}{0.0, 0.1450980392156863, 0.27450980392156865}
\definecolor{AUpantoneCyan}{rgb}{0.21568627, 0.6274509, 0.79607843}
\definecolor{AUpantoneTurkis}{rgb}{0.0, 0.6705882353, 0.6431372549}


\usepackage[
    paper=b5paper,
    inner=17.5mm,         % Inner margin % for book 20mm;   % for report 20mm
    outer=17.5mm,         % Outer margin
    % bindingoffset=10mm, % Binding offset FOR BOOK PRINTING
    top=20mm,           % Top margin
    bottom=20mm,        % Bottom margin
]{geometry}

\usepackage{tgpagella} % text only for Palatino font
\usepackage{mathpazo}  % math & text for Palatino font
\usepackage{titlesec}
% For chapter format
\titleformat{\chapter}[block]
  {\normalfont\LARGE}
  {\normalfont\makebox[\textwidth][r]{\makebox[\textwidth][r]{\scalebox{3}{\thechapter}}}}
  {10pt}
  {\vspace{1ex}\filright}
  [\vspace{1ex}\titlerule]
\titlespacing*{\chapter}{0pt}{10pt}{20pt}
\renewcommand{\thechapter}{\Roman{chapter}}  % to have roman numerals in toc
% For title of sections format
\titleformat*{\section}{\Large}
\titleformat*{\subsection}{\large}
\titleformat*{\subsubsection}{\normalsize}
\titleformat*{\paragraph}{\scshape}
\titleformat*{\subparagraph}{\scshape}
\usepackage{emptypage}  % To remove headers on empty pages
\usepackage{fancyhdr}  % to make headers and footers
\pagestyle{fancy}  % Enable the headers specified in this block and page style
\fancyhf{}  % clear headers and footers
\fancyhf[RFO, LFE]{\thepage}  % page number on the right on odd pages and the left of even pages
\fancyhf[RHO]{\leftmark}  % chapter name in the right of odd pages
\fancyhf[LHE]{\rightmark}  % section title in the left of even pages
\renewcommand{\headrulewidth}{0pt} % rule width

% Redefine the plain page style for chapter page
\fancypagestyle{plain}{%
  \fancyhf{}%
  \fancyhf[RFO, LFE]{\thepage}  % page number on the right on odd pages and the left of even pages
  \renewcommand{\headrulewidth}{0pt}% Line at the header invisible
  \renewcommand{\footrulewidth}{0.0pt}% Line at the footer invisible
}

% To change the math font to standard LaTeX Computer Modern, comment to use Euler font
\SetSymbolFont{operators}   {normal}{OT1}{cmr} {m}{n}
\SetSymbolFont{letters}     {normal}{OML}{cmm} {m}{it}
\SetSymbolFont{symbols}     {normal}{OMS}{cmsy}{m}{n}
\SetSymbolFont{largesymbols}{normal}{OMX}{cmex}{m}{n}
\SetSymbolFont{operators}   {bold}  {OT1}{cmr} {bx}{n}
\SetSymbolFont{letters}     {bold}  {OML}{cmm} {b}{it}
\SetSymbolFont{symbols}     {bold}  {OMS}{cmsy}{b}{n}
\SetSymbolFont{largesymbols}{bold}  {OMX}{cmex}{m}{n}

\SetMathAlphabet{\mathbf}{normal}{OT1}{cmr}{bx}{n}
\SetMathAlphabet{\mathsf}{normal}{OT1}{cmss}{m}{n}
\SetMathAlphabet{\mathit}{normal}{OT1}{cmr}{m}{it}
\SetMathAlphabet{\mathtt}{normal}{OT1}{cmtt}{m}{n}
\SetMathAlphabet{\mathbf}{bold}  {OT1}{cmr}{bx}{n}
\SetMathAlphabet{\mathsf}{bold}  {OT1}{cmss}{bx}{n}
\SetMathAlphabet{\mathit}{bold}  {OT1}{cmr}{bx}{it}
\SetMathAlphabet{\mathtt}{bold}  {OT1}{cmtt}{m}{n}

% Used added packages
\usepackage[nolist,nohyperlinks]{acronym}
\usepackage[shortlabels]{enumitem}
% \usepackage{blindtext}European
\usepackage[export]{adjustbox}
\usepackage{wrapfig, siunitx, caption, subcaption, multirow, nicefrac}
\usepackage{color, soul}
\usepackage{makecell}
\usepackage{pdfpages}
\usepackage{rotating}
\usepackage[english]{babel}
\usepackage{mathtools,amsmath, amssymb, amsthm} % For including math equations, theorems, symbols, etc
%to write big norm brackets in equations
\newcommand\norm[1]{\left\lVert#1\right\rVert}
\newcommand\normx[1]{\left\Vert#1\right\Vert}
\usepackage{hyperref}
\usepackage{pifont}
\newcommand{\cmark}{\ding{51}}%
\newcommand{\xmark}{\ding{55}}%
\usepackage[T1]{fontenc} % Use 8-bit encoding that has 256 glyphs
\usepackage[utf8]{inputenc} % Required for including letters with accents
\usepackage{graphicx} % Required for including images
\graphicspath{{figs/}} % Set the default folder for images
\usepackage{enumitem} % Required for manipulating the whitespace between and within lists
\usepackage{lipsum} % Used for inserting dummy 'Lorem ipsum' text into the template

\DeclareMathOperator*{\argmax}{arg\,max}
\DeclareMathOperator*{\argmin}{arg\,min}
\usepackage{varioref} % More descriptive referencing
\usepackage{listings}
\usepackage{epstopdf}
\usepackage[symbol]{footmisc}  % to add footmarks as symbols
\renewcommand{\thefootnote}{\fnsymbol{footnote}}
\usepackage[some]{background}  % to add watermark images to some pages
% Defininf here the bibliography style and format. Sort&compress stands for compressed formatting e.g. [3-7]
\usepackage[square, numbers, sort&compress]{natbib}
\usepackage{float}
\usepackage{booktabs}
\usepackage{microtype}
% to have the name of figures, tables, and equations based on chapters or not
% \counterwithout{figure}{chapter}
% \counterwithout{table}{chapter}
% \counterwithout{equation}{chapter}

\usepackage{cleveref}

\usepackage{nomencl}
\makenomenclature

\newenvironment{dedication}
  {\clearpage           % we want a new page
   \thispagestyle{empty}% no header and footer
   \vspace*{\stretch{1}}% some space at the top 
   \itshape             % the text is in italics
   \raggedleft          % flush to the right margin
  }
  {\par % end the paragraph
   \vspace{\stretch{3}} % space at the bottom is three times that at the top (to make it pretty)
   \clearpage           % finish off the page
  }

% For book class to not overlap roman numerals in toc
\usepackage{tocloft}
\setlength{\cftchapnumwidth}{3em}
\setlength{\cftsecnumwidth}{3em}
\setlength{\cftsubsecnumwidth}{4em}
\renewcommand\cftaftertoctitle{\vskip10pt\par\noindent\hrulefill\par\vskip-30pt}  % to add a horizontal line below toc title
\renewcommand\cftafterloftitle{\vskip10pt\par\noindent\hrulefill\par\vskip-30pt}  % to add a horizontal line below toc title
\renewcommand\cftafterlottitle{\vskip10pt\par\noindent\hrulefill\par\vskip-30pt}  % to add a horizontal line below toc title

% to make the spacing in the table of figures and tables consistent
% vertical
\renewcommand\cftfigafterpnum{\vskip5pt\par}
\renewcommand\cfttabafterpnum{\vskip5pt\par}
% horizontal
\cftsetindents{figure}{0em}{3.5em}
\cftsetindents{table}{0em}{3.5em}

%---------------------------------------------------------------------------------------
%     CODE
%---------------------------------------------------------------------------------------
% Set up caption font size
\captionsetup[figure]{font=small,labelfont=bf}
\captionsetup[table]{font=small,labelfont=bf}

\lstdefinestyle{mystyle}{
    basicstyle=\footnotesize,
    breakatwhitespace=true,         
    breaklines=true,                 
    captionpos=b,                    
    keepspaces=true,                 
    numbers=left,                    
    numbersep=5pt,                  
    showspaces=false,                
    showstringspaces=false,
    showtabs=false,                  
    tabsize=2
}
\lstset{style=mystyle}

%----------------------------------------------------------------------------------------
%	THEOREM STYLES
%---------------------------------------------------------------------------------------

\theoremstyle{definition} % Define theorem styles here based on the definition style (used for definitions and examples)
\newtheorem{definition}{Definition}

\theoremstyle{plain} % Define theorem styles here based on the plain style (used for theorems, lemmas, propositions)
\newtheorem{theorem}{Theorem}

\theoremstyle{remark} % Define theorem styles here based on the remark style (used for remarks and notes)

%----------------------------------------------------------------------------------------
%	HYPERLINKS
%---------------------------------------------------------------------------------------

\hypersetup{
%draft, % Uncomment to remove all links (useful for printing in black and white)
colorlinks=true, breaklinks=true, bookmarksnumbered,
urlcolor=AUpantone, linkcolor=AUpantone, citecolor=AUpantone, % Link colors
pdftitle={Vision-based navigation for underwater safety-critical applications}, % PDF title
pdfauthor={Olaya Álvarez Tuñón \textcopyright}, % PDF Author
pdfsubject={Fluid mechanics, Optimisation, Turbulence modelling}, % PDF Subject
pdfkeywords={PhD thesis, Fluid mechanics, Optimisation, Turbulence modelling}, % PDF Keywords
pdfcreator={pdfLaTeX}, % PDF Creator
pdfproducer={LaTeX} % PDF producer
}
% Other functions
\newcommand{\vect}[1]{\mathbf{#1}} %In order to show vectors in bold
\usepackage{appendix}  % to make appendices
% C++ typeset in appendix
\usepackage{listings}
% Set the default code style
\lstset{
    % frame=tb, % draw a frame at the top and bottom of the code block
    tabsize=4, % tab space width
    showstringspaces=false, % don't mark spaces in strings
    numbers=left, % display line numbers on the left
    commentstyle=\color{green}, % comment color
    keywordstyle=\color{blue}, % keyword color
    stringstyle=\color{red} % string color
}

%%% some macros to ease writing equations and such
\newcommand{\volsym}{\rlap{\kern.08em--}V} % Volume symbol
\newcommand\lrp[1]{\left( #1 \right)}
\newcommand\lrb[1]{\left[ #1 \right]}
\newcommand\lrcb[1]{\left\{ #1 \right\}}
\newcommand\avg[1]{\left\langle #1 \right\rangle}

%%% Extra math macros
\newcommand{\Partialdiff}[2]
    {\frac{\partial #1}{\partial #2}}
    
\def\Pardt{\partial_t}
\def\Pardi{\partial_i}
\def\Pardj{\partial_j}
    
\newcommand{\diff}[2]
    {\frac{d #1}{d #2}}
    
\newcommand{\mean}[1]
    {\langle #1 \rangle}

\def\ij{{ij}}

% Final touches
% \newcommand\customfont[1]{{\usefont{T1}{fonts/AUPassata_Light}{m}{n} #1 }}  % To use custom AU font (font files under "fonts" directory. Need to use XeLaTeX compiler)

\backgroundsetup{contents=\includegraphics{figs/formal/watermarkOddPage.pdf}, scale=1, opacity=0.05, angle=0}  % define the watermark figure for odd pages

\hyphenation{Fortran hy-phen-ation} % Specify custom hyphenation points in words with dashes where you would like hyphenation to occur, or alternatively, don't put any dashes in a word to stop hyphenation altogether


\newcommand{\HRule}{\rule{.9\linewidth}{.6pt}} % New command to make the lines in the title page
\definecolor{greenEIVA}{HTML}{CADB2A}
\definecolor{redEIVA}{HTML}{FF7043}
\definecolor{greenEIVAdark}{HTML}{749C48} 

%%%%%%%%%%%%%%%%%%%%%