\subsection{Evaluation Criteria}
The evaluation framework directly aligns with the research goals through quantitative metrics assessing each component's contribution to the core hypothesis.}

\begin{table}[h!]
\centering
\caption{Performance Metrics for Hypothesis Validation}
\label{tab:eval_metrics}
\footnotesize
\begin{tabular}{p{0.22\textwidth}p{0.73\textwidth}}
\toprule
\textbf{Component} & \textbf{Evaluation Metrics} \\
\midrule

\textbf{Tether-Aware Planning} &
\begin{itemize}[leftmargin=*,noitemsep]
\item \textit{Safety}: Max allowable tether length exceedance time
\item \textit{Coverage}: Completeness (\% inspected area) on complex structures
\item \textit{Computation time}
\end{itemize} \\

\textbf{Perception-Aware Control} &
\begin{itemize}[leftmargin=*,noitemsep]
\item \textit{Safety}: Distance from inspection surface
\item \textit{Inspection Quality}: How centered the surfaces being inspected (defined in paper). 
\item \textit{Robustness}: Performance under operational disturbances
\end{itemize} \\

\textbf{Data-Driven Modelling} &
\begin{itemize}[leftmargin=*,noitemsep]
\item  \textit{Modelling Error}: Mean squared error in predicting environmental disturbances
\item \textit{Trajectory Tracking Error}:  Mean squared error in trajectory tracking
\end{itemize} \\



\bottomrule
\end{tabular}
\end{table}
