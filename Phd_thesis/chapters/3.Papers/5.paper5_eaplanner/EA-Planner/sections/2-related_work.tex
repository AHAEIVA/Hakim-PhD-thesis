\section{State of the art}
\label{sec:related_work}

% Paragraph 1: Introduction to Tether Importance and Entanglement Definition
Tethers play a vital role in numerous mobile robotics applications, offering crucial power delivery and reliable data transmission. It enables extended operational periods and ensures consistent communication links for the robot. However, the risk of entanglement with obstacles represents a significant planning challenge. To systematically address this, a taxonomy of entanglement definitions has been presented in \cite{definitions}, cataloging existing interpretations from the literature and introducing new definitions, paving the way for developing new path planning strategies that account for entanglement.

% Paragraph 2: Early Approaches - 2D Offline Path Planning (Single Robot)
Initial efforts to develop entaglement-aware path planners often focused on two-dimensional environments and employed offline computation strategies. For instance, \cite{rov_mccammon} and \cite{mechsy2017novel} proposed methods for planning paths in 2D to cover a predefined set of waypoints while considering tether constraints. Notably, \cite{mechsy2017novel} formulated the path planning problem for a \ac{ROV} as a mixed-integer programming problem. Their approach first solves a \ac{TSP} to find an optimal waypoint sequence and then incorporates homotopic constraints during path generation to minimize the likelihood of tether entanglement. While effective for predefined scenarios, these offline methods lack the adaptability required for dynamic environments.

% Paragraph 3: Advancements - 2D Online Path Planning (Single Robot)
Addressing the need for real-time adaptability, subsequent research explored online path planning algorithms, still primarily in 2D. \cite{kim} introduced a homotopy-augmented topological approach combined with graph search techniques, allowing for dynamic adjustments to the path based on environmental perception. Similarly, \cite{withy} developed a hybrid A* variant that utilizes a modified tangent graph. This method efficiently plans curvature-constrained paths for tethered robots subject to winding angle constraints, demonstrating guarantees and providing simulation results for online entanglement avoidance. %These online planners represent a significant step towards real-time tether management.

% Paragraph 4: Scaling Complexity - Multi-Robot Systems in 2D
The complexity of tether management increases significantly when coordinating multiple robots. Several approaches have tackled this challenge in 2D. Early work by \cite{hert1996ties} laid groundwork in this area, followed by methods like \cite{zhang2019planning}. More recently, \cite{cao2023neptune} presented an efficient online path planner for multi-robot systems. Their method integrates a homotopy-based high-level planner with trajectory optimization and smoothing techniques to generate entanglement-free paths. However, despite its online capability, this approach remains constrained to 2D environments.

% Paragraph 5: Moving to Three Dimensions - 3D Path Planning (Single Robot)
Real-world applications frequently require navigation in 3D environments, such as with underwater robots. Consequently, work by  \cite{bhattacharya2012topological} and \cite{martinez2021optimization} explored topological aspects and optimization techniques for 3D tethered navigation. \cite{petit2022tape} specifically presented a 3D exploration path planner that incorporates explicit contact avoidance constraints for the tether, facilitating safer navigation for single tethered robots in complex three-dimensional spaces.

% Paragraph 6: 3D Multi-Robot Path Planning
 \cite{hert1999motion} extended their earlier 2D work to handle the increased complexity of 3D multi-robot scenarios. Further advancements by \cite{patil2023coordinating} and  \cite{cao2023path} introduced path planning strategies that explicitly consider the topological constraints imposed by multiple interacting tethers in 3D. While these methods advance the state-of-the-art in multi-robot coordination, they are generally designed for offline computation and are not suited for online path planning where real-time implementation is essential.

% Paragraph 7: Identifying the Research Gap
% Summarize the limitations of existing work: 
In summary, existing tether-aware planners often face limitations for practical online coverage path planning (CPP) in complex 3D settings. Many are too computationally intensive for real-time use \cite{mechsy2017novel, hert1999motion, patil2023coordinating, cao2023path}, lack integrated tether-aware CPP frameworks, or rely on simplifying assumptions like 2D environments or basic obstacle shapes \cite{kim, withy, cao2023neptune}, hindering generalization to real-world inspection tasks.
% Paragraph 8: Introducing the Proposed Solution (REACT) based on its key contributions
To address these limitations, we propose \ac{REACT}, a novel approach that enables real-time, entanglement-aware path planning in arbitrary 3-D environments.


% --- End of Literature Review Section ---













































% % Introduce tether in literature
% Tether considerations in autonomous robot planning has drawn increasingly attention in robotics literature due to their critical role in enabling robots to operate autonomously in unknown environments for a wide range of mobile robots. 
% % Entaglement definitions
% A taxonomy of entanglement definitions is presented in \cite{definitions} which catalogs existing definitions from the literature and also introduces six new definitions.  These definitions open the door to the design of new entanglement aware path planning approaches.

% % 2-D , offline
% Regarding the path planners, many path planning methods have been proposed to address tether entanglement in 2-D environments. \cite{rov_mccammon} and \cite{mechsy2017novel} propose 2-D offline path planning to cover a set a way-points. In particular, \cite{mechsy2017novel} formulate the path planning problem for a \ac{ROV} as a mixed-integer programming problem. Their method solves a \ac{TSP} to determine the optimal sequence of waypoint visits, while incorporating homotopic constraints to reduce the likelihood of tether entanglement.
% % 2-D , online
% Also in 2-D, but in an online fashion, a homotopy-augmented topological approach, combined with graph search techniques is proposed by \cite{kim}. In addition to offline methods, online path planners allow real-time entanglement avoidance \cite{kim}, \cite{withy}. A hybrid A* variant uses a modified tangent graph to efficiently plan curvature-constrained, tethered robot paths under winding angle constraints, with guarantees and simulation results provided \cite{withy}.
% % 2-D , Multirobot
% Other approaches, also in 2-D address the tether constraints problem for the multi-robot case in \cite{zhang2019planning}, \cite{hert1996ties}, and \cite{cao2023neptune}. An efficient online path planner is presented in \cite{cao2023neptune}, combining a homotopy-based high-level planner with trajectory optimization and smoothing to generate entanglement-free paths for multi-robot systems. However, this approach is limited to 2-D environments. 
% % 3-D , Multirobot
% Entanglement avoidance has also been investigated in 3-D environments, as demonstrated in \cite{petit2022tape}, \cite{martinez2021optimization}, and \cite{bhattacharya2012topological}. In particular, \cite{petit2022tape} presents a 3-D exploration path planner that incorporates contact avoidance constraints on the tether, enabling safe navigation for tethered robots in three-dimensional spaces. For multi-robot operations, \cite{hert1999motion} extends earlier work to handle the increased complexity of coordinating multiple tethered robots. Further advancements by \cite{patil2023coordinating} and \cite{cao2023path} introduce improved coordination strategies that consider topological constraints imposed by multiple tethers. However, these approaches are not designed for real-time, online path planning.

% % NICHE and summary of research gap 
% In summary, most existing planners suffer from limitations that hinder their use in practical, online coverage path planning applications. Many are too computationally intensive for real-time implementation, and there is a lack of coverage path planning frameworks that explicitly account for the tether during planning. Existing tether-aware methods often rely on simplifying assumptions, such as 2-D environments or simplistic obstacle shapes (e.g., circles or cylinders), and thus fail to generalize to complex, real-world environments—particularly in inspection tasks.
% % Introduce REACT
% To address these limitations, we propose \ac{REACT}, a novel approach that enables real-time, entanglement-aware path planning in arbitrary 3-D environments.

