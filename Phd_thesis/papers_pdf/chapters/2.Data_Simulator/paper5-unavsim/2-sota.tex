\section{State-of-the-art} \label{sec:soa}



Robotics simulation tools have significantly advanced in recent years, with a focus on providing  high-fidelity and photorealistic visual rendering. IsaacSim \cite{simulator:isaacsim}, developed by Nvidia, is a recent example that includes both high-fidelity contact simulation and high-quality image rendering provided by Omniverse, making it suitable for simulating robotic grippers and walking robots. Another example is Microsoft's AirSim \cite{rw:shah2018airsim}, yet another popular robotics simulator specifically designed for aerial vehicles. AirSim utilizes its Fastphysics engine for physics simulation and \ac{UE4} for visualization.

While progress in robotics simulation tools has been rapid, underwater robotics simulation tools have lagged behind. UWSim \cite{uwsim} and UUV Simulator \cite{uuv} are the two most commonly used underwater simulators \cite{simulator:survey}; however, they are now discontinued. %\HAK{What does this mean? not supported by the developers? or not supported by the new versions of operating systems? can you please write your claim clearer?}
A more recent simulator, DAVE \cite{projectdave}, was developed as a more modern version of the UUV simulator that supports more \ac{ROV} models and underwater sensors. However, the aforementioned simulators are based on Gazebo, which has the disadvantage of unrealistic rendering. This limits their usefulness for training and testing \ac{AI} algorithms that often rely on image inputs. To address this issue, HoloOcean \cite{holoocean} was developed using \ac{UE4} for rendering and written in Python, but it lacks support for \ac{ROS} \cite{ros}. Another example is MARUS \cite{marus}, which has not yet released its open-source implementation. The simulator uses Unity3D for visualization and integrates with \ac{ROS}. However, it lacks support for essential robotics and \ac{AI} tools, such as commercial autopilots \cite{px4}, or OpenAI's Gym environments \cite{simulator:openai_gym}, which are important tools for developing \ac{AI} and control algorithms for autonomous vehicles.
%
\begin{table*}[t]
%\caption{Marine robotics simulators comparison. UNav-Sim has superior rendering quality and versatility in supporting robotics tools, making it the best candidate for vision-based robotics applications and \ac{AI} research.  }
\caption{Marine robotics simulators comparison showing UNav-Sim's superior rendering quality and versatility.}
\centering
\footnotesize

\begin{tabular}{ cccccc}


\toprule

%\rowcolor{gray!20}

Simulator & Year & Rendering  quality &ROS Support& Autopilot & OS 
\\
\midrule

UWSim \cite{uwsim} & 2012 &Low& ROS 1 & None & Linux \\

UUV \cite{uuv} &2016 & Low& ROS 1 & Ardupilot & Linux 
 \\

URSim \cite{simulator:ursim} & 2019 & Moderate & ROS 1 & N/A& Linux \\

HoloOcean \cite{holoocean} & 2022 & High  & N/A& N/A  & Linux/Windows  \\

DAVE \cite{projectdave}& 2022 &Low &  ROS 1& PX4/Ardupilot  &Linux  \\

MARUS \cite{marus} & 2022 & Moderate & ROS 1,2 & N/A & Linux/Windows
  \\
\midrule

\textbf{UNav-Sim (Ours)} &2023 & Highest & ROS 1,2 & PX4/Ardupilot &  Linux/Windows \\
\bottomrule

\label{Tab:comparision}

\end{tabular}


\label{Tab:comparision2}
\end{table*}
%

A comparison of the capabilities of various open-source underwater robotics simulators, including the proposed simulator, is presented in Table \ref{Tab:comparision}. Amongst all simulators evaluated, the present work, UNav-Sim, stands out for its superior rendering quality, achieved through the utilization of the \ac{UE5} graphics engine. Additionally, UNav-Sim supports a range of tools commonly used in developing robotics solutions, such as \ac{ROS}, gym environments, and autopilot systems. Furthermore, UNav-Sim is compatible with both Windows and Linux operating systems.

