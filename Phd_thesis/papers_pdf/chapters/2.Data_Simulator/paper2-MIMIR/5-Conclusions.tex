\section{Conclusions and future work} 
\label{mimir:sec:conclusions}

This work presents a multipurpose synthetic underwater dataset for developing computer vision algorithms such as SLAM, segmentation, and depth estimation. 
We implement delentropy and motion diversity metrics to assess the information contained in the dataset, which are used to compare MIMIR-UW with other datasets. The metrics show that, in comparison to other datasets, the images in MIMIR-UW present more challenging conditions due to dark and blurry images.
The low-textured images and high parallax lead to failure and drift in visual SLAM algorithms. 
Similar conditions also hinder segmentation and depth estimation methods. To the best of our knowledge, MIMIR-UW introduces new challenges for the methods evaluated in this paper, thus opening room for the development of more robust and generalizable methods. Moreover, the potential of MIMIR for sim-to-real transfer in segmentation and depth estimation has been demonstrated.

Future work includes generating more diverse underwater conditions, rotation motions, and annotations for segmentation and optical flow, all within a complete underwater simulation framework that integrates underwater physics.
%Optical Flow + Aim for more photorealism/photorealistic scenes generated from real images