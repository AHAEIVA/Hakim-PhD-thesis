%\subsection{\ac{MPC} Challenges in Marine Robotics Autonomous Inspection Tasks}
%\label{subsec:mpc_barriers}

While \ac{MPC} improves autonomous capabilities, there are fundamental challenges that could hinder \ac{MPC}'s performance and deployment in marine environments, which include:

\textbf{1. Simulation-Reality Discrepancy:} Existing underwater simulators lack both \textit{visual realism} (critical for perception algorithms) and \textit{physical fidelity} in modeling hydrodynamic interactions, making pre-deployment controller validation unreliable. This gap forces costly trial-and-error field testing.

\textbf{2. Environmental Uncertainty:} Marine disturbances—waves, currents, —exhibit multi-scale spatiotemporal variations that defy analytical modeling. Traditional \ac{MPC} performance degrades rapidly when hydrodynamic predictions from reality.

\textbf{3. Handling Tether Constraints:} Simultaneously optimizing inspection coverage while avoiding tether entanglement remains a significant challenge. The tether is inherently difficult to model accurately, and its nonlinear, dynamic behavior is hard to incorporate effectively within MPC frameworks. 

\textbf{4. Perception-Control Decoupling:} Most current approaches treat visual inspection and trajectory tracking as separate objectives. This decoupling often leads to trade-offs—sacrificing coverage quality for path accuracy or vice versa—resulting in reduced robustness to environmental disturbances and system uncertainties.


Despite the potential of \ac{MPC} to enhance autonomous inspection in marine robotics, key limitations persist due to modeling inaccuracies, environmental uncertainties, and system-level decoupling between perception and control. These challenges highlight the need for more robust, adaptive control strategies that can handle real-world disturbances, incorporate perception-driven feedback, and respect complex physical constraints such as tether dynamics. 

In the following section, we propose a research hypothesis aimed at addressing these limitations by integrating entanglement-aware planners, a visual tracking-based cost function, and data-driven models into the \ac{MPC} framework. Additionally, we leverage advanced simulation environments to enable more reliable and effective autonomous inspection in complex and uncertain marine settings.
