\phantomsection
%\pdfbookmark[0]{Titelblad}{titelblad}

% \vspace{10pt}

\chapter*{Preface}
\addcontentsline{toc}{chapter}{Preface}
% \section*{Project Design} Innovative Training Networks (ITN) - Marie Curie Actions
This thesis concludes a three-year research journey carried out between July 2022 and June 2025. This work was conducted out as part of the Ph.D. program at the Graduate School of Technical Sciences (GSTS) at Aarhus University, in collaboration with EIVA A/S in Denmark. 

Aarhus University served as the primary academic institution for this project, with the majority of the research conducted within the Department of Electrical and Computer Engineering, specifically in the Artificial Intelligence in Robotics group. This work was carried out under the supervision of Professor Erdal Kayacan. Industrial supervision was provided by Dr. Yury Brodskiy, research lead in the Autonomy department at EIVA A/S. 

The project was further enriched by the collaboration and mentorship of Dr. Mohit Mehindratta from GIM Robotics, whose expertise and ongoing guidance had a significant impact on the project's direction and outcomes. Professor Andriy Sarbakha from Aarhus University also contributed critical insights and guidance. In addition, during my research stay at the the German Research Center for Artificial Intelligence (DFKI), Dr. Bilal Wehbe provided thoughtful input and technical support that helped conduct the experimental aspects of the work.

This thesis focuses on the development of data-driven optimal control and navigation methods for autonomous marine robotics inspection applications. The outcomes of this project include six published scientific papers in the fields of control theory and robotics and one submitted journal paper under review. The author of this thesis is the main author of six of these publications.


%The thesis is structured as a collection of articles, comprising an introduction to the research topics and state-of-the-art developments, a detailed description of the proposed research questions, the most relevant manuscripts produced during the Ph.D. project, and concluding remarks.

\section*{Recognition of funding and grants}
\addcontentsline{toc}{section}{Recognition of funding and grants}
This research receives support from Innovation Fund Denmark grant number 2040-00032B and EIVA a/s.

\section*{Reading guide}
\addcontentsline{toc}{section}{Reading guide}
The contents of the individual Chapters in this dissertation with published manuscripts are introduced with an abstract summarising the contents of the Chapter and a reference to the original published work. References throughout the dissertation are listed at the end of the main document. The standard numbered style denounces the references. Hence, a citation is referred to by [Number]. Bibliographical citations appear at the end of the document where journal and proceedings papers are referred by author, article title between quotes, and journal names in \textit{italics}. Books are referred by author, title in \textit{italics}, publisher, edition, and year, while websites are referred by author, title, year, URL, and time of last visit. Figures, tables, and equations are numbered according to the particular Chapter they are placed in, where uppercase Roman numerals number Chapters. Therefore, the first figure in Chapter Three is assigned with figure number III.1 and the second III.2, etc. Descriptive captions for tables are found above relevant tables, and captions for figures are found under relevant figures. Additional explanatory concepts are added with footnotes, labeled with $\ast$, $\dagger$, $\ddagger$, and so forth.

\subsection*{Chapter's Guide}
This thesis follows a collection-of-papers format and is organized into two main parts. 

\textbf{Part I} provides a comprehensive overview and contextual foundation for the research. It begins with a general introduction to the problem domain, followed by the formulation of the research hypothesis and an outline of the thesis structure. A background chapter on Model Predictive Control (\ac{MPC}) is included to support readers unfamiliar with the control framework. The core contributions of the thesis are presented across five dedicated chapters, each corresponding to one of the included peer-reviewed publications. These chapters provide contextual framing and technical insight, and in some cases, incorporate updated figures or results from post-publication experiments. Part I concludes with a final chapter that summarizes the main findings, discusses their implications, and outlines potential directions for future research in autonomous marine robotics and control.

\textbf{Part II} includes the full versions of the four published (or accepted) research papers that constitute the core scientific contributions of the thesis. These are presented in their original form, with only minor formatting adaptations to align with the thesis layout.







\section*{List of Publications}
\addcontentsline{toc}{section}{List of Publications}
The following scientific manuscripts (in chronological order) have been published, accepted, or are under review at the time of submission of this thesis:

\vspace{1em}

\tcbset{
  colback=gray!5!white, 
  colframe=gray!75!black, 
  boxrule=0.5pt, 
  arc=4pt, 
  boxsep=5pt, 
  left=5pt, 
  right=5pt, 
  top=5pt, 
  bottom=5pt,
}

%\begin{tcolorbox}[colback=gray!10, colframe=gray!80, title=\textbf{Publications}]
\textbf{1.} \textbf{Amer, A.}, Mehndiratta, M., Brodskiy, Y., \& Kayacan, E. (2025). \textit{REACT: Real-time Entanglement-Aware Coverage Path Planning for Tethered Underwater Vehicles}. Submitted to the IEEE Transactions on Field Robotics, under review.\\[0.8em]

\textbf{2.} \textbf{Amer, A.}, Falsegar, D., Brodskiy, Y., \& Sarabakha, A. (2025). \textit{Modelling of Underwater Vehicles using Physics-Informed Neural Networks with Control}. Submitted to the International Joint Conference on Neural Networks (IJCNN), Accepted for publication.\\[0.8em]

\textbf{3.} \href{https://ieeexplore.ieee.org/document/10916556/authors#authors}{\textbf{Amer, A.}, Mehndiratta, M., Brodskiy, Y., \& Kayacan, E. (2025). \textit{Empowering Autonomous Underwater Vehicles Using Learning-based Model Predictive Control With Dynamic Forgetting Gaussian Processes}. \textit{IEEE Transactions on Control Systems Technology}.}\\[0.8em]

\textbf{4.} \href{https://ieeexplore.ieee.org/document/10900576}{Liang, W., \textbf{Amer, A.}, Mehndiratta, M., Chen, Z., Yao, B., \& Kayacan, E. (2025). \textit{Adaptive Robust Control Integrated With Gaussian Processes for Quadrotors: Enhanced Accuracy, Fault Tolerance and Anti-Disturbance}. \textit{IEEE Transactions on Systems, Man, and Cybernetics: Systems}.}\\[0.8em]

\textbf{5.} \href{https://ieeexplore.ieee.org/document/10406329}{\textbf{Amer, A.}, Mehndiratta, M., Sejersen, J.L.F., Pham, H.X., \& Kayacan, E. (2023). \textit{Visual Tracking Nonlinear Model Predictive Control Method for Autonomous Wind Turbine Inspection}. \textit{2023 21st International Conference on Advanced Robotics (ICAR)}, 431-438.}\\[0.8em]

\textbf{6.} \href{https://ieeexplore.ieee.org/abstract/document/10406819}{\textbf{Amer, A.}, Álvarez-Tuñón, O., Uğurlu, H.İ., Sejersen, J.L.F., Brodskiy, Y., \& Kayacan, E. (2023). \textit{UNav-Sim: A visually realistic underwater robotics simulator and synthetic data-generation framework}. \textit{2023 21st International Conference on Advanced Robotics (ICAR)}, 570-576.}\\[0.8em]

\textbf{7.} \textbf{Amer, A.}, Álvarez-Tuñón, O., Falsegar, D., Brodskiy, Y., \& Kayacan, E. (2023). \textit{MUDROV: A modular underwater defouling ROV for ship propeller cleaning}. \textit{Advanced Marine Robotics Workshop, International Conference on Intelligent Robots and Systems (IROS) 2023}.
%\end{tcolorbox}



\section*{Open-source contributions}

Additionally, as part of this PhD, several outcomes of the research have been realized as open-source projects, supporting research on learning-based control, motion planning, and simulation for field robotics applications.



\begin{itemize}
    \item \textbf{UNav-Sim} \\
    A visually realistic underwater robotics simulator and synthetic data-generation framework developed for benchmarking learning-based planning and control algorithms.\\
    \href{https://github.com/open-airlab/UNav-Sim}{\texttt{github.com/open-airlab/UNav-Sim}}

    \item \textbf{VT-NMPC: Autonomous Wind Turbine Inspection} \\
    An open-source implementation of Visual Tracking Nonlinear Model Predictive Control for autonomous drone-based wind turbine inspection.\\
    \href{https://github.com/open-airlab/VTNMPC-Autonomous-Wind-Turbine-Inspection}{\texttt{github.com/open-airlab/VTNMPC-Autonomous-Wind-Turbine-Inspection}}

    \item \textbf{Physics-Informed Neural Control (PINC)} \\
    A PyTorch-based Physics-Informed Neural Network framework for learning BlueROV dynamics. \\
    \href{https://github.com/eivacom/pinc-xyz-yaw}{\texttt{github.com/eivacom/pinc-xyz-yaw}}
\end{itemize}





\cleardoublepage

% \noindent\fbox{\parbox{\textwidth}{

\BgThispage  % add watermark

\begin{minipage}[t]{0.48\textwidth}
	\vspace*{0pt}			%\vspace*{-9pt}
	\includegraphics[width=0.74\linewidth]{figs/logos/AUpantoneBlueMPE.pdf}
\end{minipage}
\hfill
\begin{minipage}[t]{0.3\textwidth}
    \begin{flushright}
        {\small 
        \textbf{Aarhus University \\
        Department of Electrical
        and Computer Engineering} \\
        Finlandsgade 22 \\
        8200 Aarhus N, Denmark \\
        \url{http://ece.au.dk/en}}
    \end{flushright}
\end{minipage}

\vspace{5ex}

% \noindent\makebox[0.3\linewidth]{{\rule{0.25\paperwidth}{4pt}}}
\noindent\makebox[\linewidth]{{\rule{0.8\paperwidth}{4pt}}}

\vspace{5ex}

% \noindent\fbox{\parbox{\textwidth}{

\begin{minipage}[t]{0.485\textwidth}
\begin{flushleft}

% \textbf{Title:} \\[5pt]
% Robust Optimisation of Ultrasonic Flow Meters by Computational Fluid Dynamics and Enhanced Turbulence Modelling

\textbf{Thesis submitted:} \\[5pt]\hspace{2ex}
June 30\textsuperscript{st}, 2025
\\[5pt]
\textbf{Author:} \\[2pt]\hspace*{2ex}
\textbf{Abdelhakim Khaled Amer} \\\hspace*{2ex}
\href{mailto:olaya@ece.au.dk}{abdelhakim@ece.au.dk}\\\hspace*{2ex}
\href{mailto:oat@eiva.com}{abdelhakim@aucegypt.edu}\\\hspace*{2ex}

\textbf{Pages: -} \\
% \textbf{Appendices: None} \\ 
%\textbf{Project deadline: 29-5-2017}
\end{flushleft}
\end{minipage}
\hfill
\begin{minipage}[t]{0.485\textwidth}
\begin{flushleft}

\textbf{Supervisors:} \\[5pt]

\textbf{Henrik Karstoft} \\
Full Professor \\
Signal Processing and Machine learning \\
Aarhus University \\
\href{mailto:hka@ece.au.dk}{hka@ece.au.dk} \\[5pt]

\textbf{Erdal Kayacan} \\
Full Professor \\
Automatic Control Group (RAT) \\
Paderborn University \\
\href{mailto:abkar@mpe.au.dk}{erdal.kayacan@uni-paderborn.de} \\[5pt]

\textbf{Yury Brodskiy} \\
Senior Software Engineer \\
Computer Vision Group Leader \\
EIVA A/S \\
\href{mailto:ybr@eiva.com}{ybr@eiva.com} \\

\end{flushleft}
\end{minipage}
% }}

\vspace{5ex}

% \noindent\makebox[0.3\linewidth]{{\rule{0.25\paperwidth}{4pt}}}
\noindent\makebox[\linewidth]{{\rule{0.8\paperwidth}{4pt}}}


\vfill

\noindent {\footnotesize\itshape 
The contents of this Ph.D. thesis are freely accessible and distributed under the terms of the Creative Commons CC-BY-SA license, 
\begin{wrapfigure}{o}{0.1\textwidth}
    \includegraphics[width=0.1\textwidth]{figs/logos/CC_BY-SA_icon.svg.png}
\end{wrapfigure}
which permits unrestricted use, distribution, and reproduction in any medium, provided the original work is properly cited and the license type does not change. Furthermore, none of the contents of this dissertation includes plagiarism.
}

\newpage

\vspace{5ex}

{\Large \textbf{Project collaborators:}} 

\vspace{7ex}

\begin{center}

\includegraphics[width=0.48\textwidth, valign=c]{figs/logos/AUblack.pdf}
\hspace{5ex}
\includegraphics[width=0.45\textwidth, valign=c]{figs/logos/eivalogo.png}

\vspace{5ex}

\includegraphics[width=0.2\textwidth, valign=c]{Phd_thesis/figs/logos/dfki.png}
\hspace{14ex}
\includegraphics[width=0.35\textwidth, valign=c]{Phd_thesis/figs/logos/pad.png}


\vspace{2ex}

\includegraphics[width=0.25\textwidth, valign=c]{figs/logos/gim.jpg}
\hfill
\includegraphics[width=0.15\textwidth, valign=c]{Phd_thesis/figs/logos/upteko.png}
\hfill
\includegraphics[width=0.20\textwidth, valign=c]{Phd_thesis/figs/logos/airlab.jpg}


\vspace{2ex}

\includegraphics[width=0.2\textwidth, valign=c]{Phd_thesis/figs/logos/purdue.png}
\hspace{14ex}
\includegraphics[width=0.18\textwidth, valign=c]{Phd_thesis/figs/logos/zju.png}



\vspace{1ex}



\vspace{1ex}

\includegraphics[width=0.8\textwidth]{Phd_thesis/figs/logos/inno.png}
\end{center}

\cleardoublepage


