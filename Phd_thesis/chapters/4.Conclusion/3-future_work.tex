\section{Future Work and Open Questions}
\label{sec:future_work}

While this thesis presents significant advancements, it also opens avenues for further research and addresses remaining open questions in marine robotics autonomy. Building upon the contributions presented, future work should focus on the following directions:

\begin{enumerate}
    \item \textbf{Integration and Holistic Validation:}
        \begin{itemize}
            \item \textcolor{black}{Integrate the developed components into a unified framework.} For instance, utilize models learned via PINC or disturbances estimated by DF-GP within the REACT planner and the VT-NMPC controller.
            \item \textcolor{black}{Conduct extensive real-world validation} of the integrated system, particularly REACT and the marine application of VT-NMPC, in diverse and challenging underwater environments beyond controlled tank tests or simulations. This is crucial for assessing true robustness and identifying limitations not captured synthetically.
        \end{itemize}

    \item \textbf{Enhancing Simulation Capabilities (UNav-Sim):}
        \begin{itemize}
            \item Incorporate a wider range of underwater sensors (e.g., various sonar types, acoustic modems, chemical sensors) with realistic noise models.
        
           \item Enhance the accuracy of hydrodynamic modeling by making it geometry-dependent. Integrate differentiable physics capabilities to enable gradient computation for learning-based algorithms, facilitating efficient optimization and control.

        \end{itemize}

\item \textbf{Online Mapping and Adaptive Planning Integration:}
    \begin{itemize}
        \item Extend REACT to handle dynamic obstacles and support online map updates, for example by utilizing continuously updated maps instead of relying on a static environment representation.
        
        \item Investigate adaptive planning strategies that adjust coverage paths in real time based on sensor feedback and dynamically identified areas of interest during the mission.
    \end{itemize}

\item \textbf{Continual Learning and Safe Model Predictive Control:}


    \begin{itemize}
        \item Investigate methods for \textcolor{black}{online adaptation and lifelong learning} in both PINC and DF-GP frameworks, enabling continuous model refinement in response to evolving vehicle dynamics and environmental conditions during extended deployments.

        \item Explore \textcolor{black}{minimal-loss configurations} for PINC to improve computational efficiency, including the potential use of alternative architectures such as Recurrent Neural Networks (RNNs) for better temporal representation.

        \item Further develop \textcolor{black}{automated tuning of MPC weights} using black-box optimization techniques, such as Bayesian optimization or reinforcement learning, to enhance control performance with minimal manual intervention.

        \item Investigate the design of \textcolor{black}{safe terminal sets} using learning-based methods to ensure constraint satisfaction and long-term safety, while maintaining computational tractability for real-time implementation.
    \end{itemize}
