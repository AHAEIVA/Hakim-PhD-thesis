\chapter{Visually Robust Coverage Path Planning}
In this chapter, we address another critical challenge in coverage path planning: achieving visually robust inspection. We tackle this by incorporating perception objectives directly into the control framework. Specifically, we propose a novel Visual Tracking Nonlinear Model Predictive Control (VT-NMPC) framework designed to enable automated, safe, and efficient inspection of wind turbine blades using aerial robots. In contrast to traditional MPC-based approaches that track predefined position and heading trajectories, VT-NMPC introduces a surface-tracking paradigm, allowing the drone to follow inspection surfaces directly. This shift eliminates the need for precisely planned trajectories and enhances robustness to disturbances such as wind.

The proposed inspection framework consists of two key components:
\begin{itemize}
    
\item A global path planner that generates a distance-optimal sequence of surface regions to inspect across all turbine blades.

\item  A VT-NMPC controller that dynamically regulates the drone’s pose to maintain optimal distance and orientation relative to the blade surfaces, thereby ensuring high-resolution imaging while avoiding collisions.

\end{itemize}

The VT-NMPC cost function is designed to enforce visual inspection constraints — such as maintaining perpendicularity to the surface and respecting drone dynamics — making it inherently suited to the inspection task. This method reduces reliance on expert pilots, increases operational safety, and improves inspection quality.

The framework has been validated in both simulation and real-world tests on a custom-built drone platform. Results show that the proposed method achieves full blade coverage, demonstrates robustness to wind disturbances, and outperforms traditional trajectory-based MPC methods in terms of adaptability and visual inspection fidelity.

This chapter also reviews related work in automated drone-based inspection and details the complete pipeline from global path planning to real-time control. The open-sourced implementation further supports reproducibility and future research.