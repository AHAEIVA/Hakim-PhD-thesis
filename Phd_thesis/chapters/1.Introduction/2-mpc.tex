%\subsection{Optimal Control in Marine Autonomy}

%Why optimal control is needed
Optimal control has become vital in modern control systems and is key to major advancements in autonomy. The concept of formulating control and planning problems as optimization problems—where high-level behavior is defined by the user instead of a predefined sequence of actions—allows for achieving complex tasks, previously not possible with conventional control methods. 

%MPC 
Model predictive control (MPC) has emerged as a transformative technology for autonomous systems, combining three critical capabilities: (1) natural management of vehicles' highly coupled dynamics and large degrees of freedom, (2) explicit constraint handling for thrusters, operational envelopes, and collision avoidance, and (3) real-time optimization of the joint trajectory and control. Consequently, it enables capabilities and applications in marine autonomy that were previously unattainable.


%Success stories in marine robotics MPC implementation include precision docking operations for autonomous underwater vehicles (AUVs), energy-efficient station-keeping in varying current conditions, and coordinated multi-vehicle operations in dynamic environments. These applications demonstrate MPC's potential to significantly enhance operational capability and reliability in underwater missions.

    

    
    

    