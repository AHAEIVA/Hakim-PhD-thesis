%\section{Acknowledgement}
%The authors thank Adis Hodsic, Ernest Wilk, Mathias Spaniel, and Thomas Klein (EIVA) for experimental support.










\section{Conclusion}
\label{sec:conclusion}

%In this study, a learning-based \ac{MPC} framework was introduced, which leverages \ac{DF-GP} to address rapidly changing and difficult-to-model disturbances encountered in underwater robotics. 
This study introduces a learning-based \ac{MPC} framework that utilizes \ac{DF-GP} to effectively predict disturbances of varying magnitude and frequencies encountered in underwater robotics. This framework facilitates the integration of various \ac{GP} contributions, ensuring its applicability across different scenarios without the necessity for hyperparameter retuning during operation. Furthermore, the global convergence of the proposed method is analyzed. The conducted experiments demonstrate a \(59.6\%\) improvement in disturbance estimation accuracy and a \(72.9\%\) enhancement in tracking performance across various disturbance profiles. Overall, the results validate that the proposed framework outperforms its direct competitors, achieving a \(25\%\) improvement in both disturbance estimation and tracking performance. Future studies will explore the impact of dynamic input and inducing point selection. %for a fully automated learning-based \ac{MPC} framework.
