%\subsection{Optimal Control in Marine Autonomy}

%Why optimal control is needed
Trajectory optimization has become central to recent breakthroughs in autonomy, enabling robots to perform increasingly complex tasks. By formulating trajectory planning as an optimization problem—where high-level objectives are specified rather than relying on predefined sequences of actions—systems can adapt and respond intelligently to dynamic environments. At the heart of this trajectory optimization lies Model Predictive Control (MPC) which allows simultaneous optimization of trajectories and control inputs in real time. It naturally accommodates the coupled dynamics and high degrees of freedom characteristic of autonomous vehicles, while explicitly handling constraints related to thruster limitations, operational boundaries, and collision avoidance. This integration of real-time planning, dynamic feasibility, and constraint satisfaction has unlocked new capabilities in marine autonomy that were previously out of reach using conventional methods.




    

    
    

    