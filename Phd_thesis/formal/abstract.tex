\chapter*{Abstract}
\addcontentsline{toc}{chapter}{Abstract}

Marine robots have become indispensable for inspection tasks across diverse environments, from deep-sea surveying to aerial monitoring of offshore installations. Autonomy in these operations is increasingly vital—not just to reduce operational expenses, but to ensure consistent, high-quality performance with minimal human oversight. Advanced real-time control and planning algorithms such as Model Predictive Control (MPC) are key in marine robotics autonomy, as it can handle systems with high nonlinearity and many degrees of freedom. However, practical implementation of MPC in marine robotics remains in its early stages, with significant challenges still to overcome. 

This research develops optimal navigation algorithms for autonomous underwater inspection robots. It introduces novel methods that integrate learning-based modeling techniques with MPC to address the unique challenges of these environments.

A key contribution of this work is the development of an open-source marine robotics simulator that provides highly realistic rendering and physics, enabling safe, cost-effective, and efficient testing of algorithms, as well as synthetic data collection. In addition, a novel MPC-based coverage path planning approach is developed that accounts for tethering constraints while ensuring complete area coverage.

To improve the accuracy of dynamic modeling of underwater vehicles, learning-based approaches are employed to capture complex underwater dynamics across multiple timescales. In particular, Gaussian Processes and Physics-Informed Neural Networks are used to enhance the modeling of underwater vehicle dynamics, thereby enhancing MPC performance and safety.

By integrating data-driven models and task-specific path planners with MPC, this work enables the development of more reliable and safer algorithms for autonomous marine inspection. The findings advance the state-of-the-art in marine robotics and contribute towards real-world deployment of fully autonomous inspection and surveying operations in complex marine environments.



%%%%
%Arabic
\newpage
\section{تاريخ}

\setcode{utf8}

\begin{center}
    \begin{RLtext}
        \noindent\textbf{﴾ بِسْمِ ٱللَّهِ ٱلرَّحْمَـٰنِ ٱلرَّحِيمِ ﴿}
        
        \vspace{0.3cm}
        
        \noindent\textbf{﴿ وَقُلِ ٱعْمَلُوا۟ فَسَيَرَى ٱللَّهُ عَمَلَكُمْ وَرَسُولُهُۥ وَٱلْمُؤْمِنُونَ ﴾}
        
        \vspace{0.3cm}
        
        \noindent\textbf{﴾ صَدَقَ ٱللَّهُ ٱلْعَظِيمُ ﴿}
    \end{RLtext}
    \vspace{0.5cm}
\end{center}

\begin{RLtext}
أصبحت الروبوتات البحرية جزءًا لا غنى عنه في مهام التفتيش في البيئات المتنوعة، بدءًا من المسح في أعماق البحار وصولًا إلى المراقبة الجوية للمنشآت البحرية. ومع تزايد الحاجة إلى تقليل التكاليف التشغيلية وضمان الأداء العالي المستمر دون تدخل بشري، تزداد أهمية تعزيز مستوى الاستقلالية في هذه المهام. يُعد التحكم التنبؤي بالنموذج (\LR{MPC}) من أبرز الأدوات في تمكين هذا النوع من الاستقلالية، نظرًا لقدراته على التعامل مع الأنظمة ذات الطبيعة غير الخطية ودرجات الحرية العالية. ومع ذلك، لا يزال استخدامه العملي في الروبوتات البحرية في مراحله الأولى، حيث تواجهه تحديات كبيرة تتطلب حلولًا مبتكرة.

يقدم هذا البحث خوارزميات ملاحة مثلى للروبوتات البحرية الذاتية المصممة لمهام التفتيش تحت الماء. ويطرح منهجيات جديدة تجمع بين تقنيات النمذجة المعتمدة على التعلم والخوارزميات التنبؤية للتحكم، للتغلب على التحديات الخاصة بهذه البيئات.

من الإسهامات الأساسية في هذا العمل تطوير محاكي مفتوح المصدر للروبوتات البحرية يتميز بواقعية عالية من حيث الفيزياء والعرض المرئي، ما يوفر بيئة آمنة ومنخفضة التكلفة لاختبار الخوارزميات وجمع البيانات الاصطناعية. كما تم تطوير نهج جديد لتخطيط مسارات التغطية باستخدام \LR{MPC} يأخذ في الاعتبار قيود الحبل المرتبط بالروبوت، مع ضمان تغطية شاملة للمناطق المستهدفة.

ولتحسين نمذجة ديناميكيات الروبوتات تحت الماء، تم توظيف أساليب تعلمية متقدمة تُمكِّن من التقاط سلوكيات معقدة عبر أزمنة متعددة. على وجه التحديد، تم استخدام العمليات الغاوسية والشبكات العصبية المدعومة بالمعلومات الفيزيائية (\LR{PINNs}) لتعزيز دقة النماذج الديناميكية، مما يُحسن من أداء \LR{MPC} ويزيد من موثوقيته وسلامته.

من خلال دمج النماذج المعتمدة على البيانات مع مخططي المسارات المصممين خصيصًا للمهام المختلفة ضمن إطار التحكم التنبؤي، يسهم هذا العمل في تطوير خوارزميات أكثر كفاءة واعتمادية لمهام التفتيش البحري الذاتي. وتشكل النتائج المتحققة خطوة متقدمة نحو التشغيل الفعلي الكامل والمستقل لهذه الأنظمة في البيئات البحرية المعقدة.
\end{RLtext}


