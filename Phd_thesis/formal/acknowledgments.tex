\chapter*{Acknowledgments}
\thispagestyle{empty}
% This thesis would not have been possible without a fortunate and enriching sequence of events—and those who have been part of them—that have shaped my life over the past thirty years. 
% First, thank you to Juan and Carmen, my parents, who have made me the person I am today. Through your own actions, you have taught me that effort, sacrifice, kindness, and, most importantly, willingness, always pay off. Your unconditional support has been one of the main pillars of my journey here.

% Thank you to my friends from Asturias. We've been a long way together, and I feel so lucky to have you in my life.
% Another special mention goes to my friends from \textit{la Carlos III}. I treasure all the wonderful moments spent with you in the \textit{C13} (and Madrid's whereabouts). Thank you also to my former supervisor there, Carlos Balaguer, who got me into the academic career in the first place, opened up the path I have followed so far. 

% Moving to Denmark has been quite a journey, and I am very grateful for all the wonderful people I have met and all the friends I have made here. These lines go to my colleagues from Aarhus University and the Airlab. Thank you for your friendship and support. Also, I thank the people from EIVA, who have been so welcoming and have taught me a lot (and not only in the pub quizzes).
% Thank you also to Erdal and Yury, my supervisors. You have shared with me not only your time and knowledge, which have nourished my way through this thesis, but also your friendship. Our brunches, dinners, and board game nights together have had great value for me. Special mention also to Kirsten and Henrik, whose kindness towards me and my family makes me very grateful.

% This thesis here, has been supported by a Marie Curie fellowship. I owe the germ of the idea of applying for this fellowship to my former supervisor in Asturias, Rafael Corsino, who, many years ago, wisely said to a group of naive students: \textit{"The first requirement to obtain a Marie Curie fellowship is to APPLY for a Marie Curie fellowship"}. You were right! 

% Now, allow me to quote Marie Curie, who wrote:
% \textit{"You cannot hope to build a better world without improving the individuals. To that end, each of us must work for our own improvement and share a general responsibility for all humanity, our particular duty being to aid those to whom we think we can be most useful"}.
% I cannot describe how lucky and thankful I feel to have benefited from this fellowship. It aims to progress our society through, firstly, science as a whole (\textit{yeah, science!}), and secondly, through the individual progress of those working on it, as stated in Marie Curie's quote. Being part of a society following this belief is a great privilege to enjoy and a great responsibility. This thesis that you hold on your hands (or in your browser's cache) is my humble and best attempt at being worthy of such privilege. I hope you enjoy scrolling through its pages.

\begin{table}[b]
	\centering
		\begin{tabular}{c}
            \vspace{-1.1cm}
            \hspace{0.6cm}
			\includegraphics[width = 2cm]{figs/formal/signature_new.pdf}\\  % 4 cm
            \underline{\phantom{mmmmmmmmmmmmmmmmmmm}} \\
			\textit{AbdelHakim Amer} % \\
			% \href{mjrp@mpe.au.dk}{mjrp@mpe.au.dk}
		\end{tabular}
\end{table}



% \noindent Aarhus University, October 31\textsuperscript{st} 2023.

\vfill