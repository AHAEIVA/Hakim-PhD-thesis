\section{State of the art}
\label{sec:related_work}


Tether considerations in autonomous robot planning has drawn increasingly attention in robotics literature due to their critical role in enabling robots to operate autonomously in unknown environments for a wide range of mobile robots. A comprehensive survey of entanglement definitions relevant to path planning is presented in \cite{definitions}. The survey not only catalogs existing definitions from the literature but also introduces six new definitions, designed to be general and applicable to 3D navigation scenarios.  These newly proposed definitions extend the state-of-the-art by addressing scenarios involving slack tethers and non-taut configurations, which are particularly relevant in 3D navigation.   

In the literature, various 2-D path planning methods have been proposed to address the challenge of avoiding tether entanglement. \cite{rov_mccammon} and \cite{mechsy2017novel} present offline planning approaches that specifically consider the dynamics of tethered vehicles. In particular, \cite{mechsy2017novel} formulate the path planning problem for a \ac{ROV} as a mixed-integer programming model. Their method solves a \ac{TSP} to determine the optimal sequence of waypoint visits, while incorporating homotopic constraints to reduce the likelihood of tether entanglement.

A homotopy-augmented topological approach, combined with graph search techniques, addresses the entanglement planning problem \cite{kim}. In addition to offline methods, online path planners allow real-time entanglement avoidance \cite{kim}, \cite{withy}. One approach applies winding angle constraints to an A* path planner, ensuring the tether stays within user-defined limits throughout the mission \cite{withy}.

Other approaches address the tether constraints problem for the multi-robot case in \cite{zhang2019planning}, \cite{hert1996ties}, and \cite{cao2023neptune}. An efficient online path planner is presented in \cite{cao2023neptune}, combining a homotopy-based high-level planner with trajectory optimization and smoothing to generate entanglement-free paths for multi-robot systems. However, this approach is limited to 2-D environments. 

Entanglement avoidance has also been investigated in 3-D environments, as demonstrated in \cite{petit2022tape}, \cite{martinez2021optimization}, and \cite{bhattacharya2012topological}. In particular, \cite{petit2022tape} presents a 3-D exploration path planner that incorporates contact avoidance constraints on the tether, enabling safe navigation for tethered robots in three-dimensional spaces. For multi-robot operations, \cite{hert1999motion} extends earlier work to handle the increased complexity of coordinating multiple tethered robots. Further advancements by \cite{patil2023coordinating} and \cite{cao2023path} introduce improved coordination strategies that consider topological constraints imposed by multiple tethers. However, these approaches are not designed for real-time, online path planning.

In summary, most existing planners suffer from limitations that hinder their use in practical, online coverage path planning applications. Many are too computationally intensive for real-time implementation, and there is a lack of coverage path planning frameworks that explicitly account for the tether during planning. Existing tether-aware methods often rely on simplifying assumptions, such as 2-D environments or simplistic obstacle shapes (e.g., circles or cylinders), and thus fail to generalize to complex, real-world environments—particularly in inspection tasks.

To address these limitations, we propose \ac{REACT}, a novel approach that enables real-time, entanglement-aware path planning in arbitrary 3-D environments.

%% Niche why our method is new?













%The aforementioned methods treat control and planning as separate problems, typically employing a search-based path planner alongside a low-level controller. However, the coupling between these two problems is not addressed using approaches such as receding horizon control or \ac{MPC}. Additionally, these methods consider only the current state, without accounting for future entanglement, resulting in reactive rather than predictive behavior. A continuous entanglement metric is also lacking, as existing approaches generally classify states as either entangled or non-entangled, failing to account for partial entanglement. Furthermore, most approaches have limited integration with environmental uncertainty, particularly in partially mapped or changing environments where tether behavior becomes less predictable. Computational efficiency remains a challenge for many sophisticated tether-aware planners, constraining their applicability in resource-limited platforms or real-time applications.Our work addresses these limitations by introducing a novel approach that provides a continuous metric for quantifying entanglement severity, integrates planning and control through MPC framework, and adopts a predictive strategy for tether management. This approach enables more nuanced decision-making in complex environments, allowing robots to select paths that minimize entanglement rather than simply avoiding it completelyentangled, failing to account for partial entanglement.



